%%%%%%%%%%%%%%%%%%%%%%%%%%%%%%%%%%%%%%%%%
%
% Author: Fabrizio Zeni
%
%%%%%%%%%%%%%%%%%%%%%%%%%%%%%%%%%%%%%%%%%

%----------------------------------------------------------------------------------------
%	PACKAGES AND OTHER DOCUMENT CONFIGURATIONS
%----------------------------------------------------------------------------------------

\documentclass{article}

\usepackage{fancyhdr} % Required for custom headers
\usepackage{lastpage} % Required to determine the last page for the footer
\usepackage{extramarks} % Required for headers and footers
\usepackage[usenames,dvipsnames]{color} % Required for custom colors
\usepackage{graphicx} % Required to insert images
\usepackage{listings} % Required for insertion of code
\usepackage{courier} % Required for the courier font
\usepackage{lipsum} % Used for inserting dummy 'Lorem ipsum' text into the template
\usepackage{caption}


\definecolor{mygreen}{rgb}{0,0.6,0}
\definecolor{mygray}{rgb}{0.5,0.5,0.5}
\definecolor{mymauve}{rgb}{0.58,0,0.82}


% Margins
\topmargin=-0.45in
\evensidemargin=0in
\oddsidemargin=0in
\textwidth=6.5in
\textheight=9.0in
\headsep=0.25in

\linespread{1.1} % Line spacing

% Set up the header and footer
\pagestyle{fancy}
\lhead{\hmwkAuthorName} % Top left header
\chead{\hmwkClass\ : \hmwkTitle} % Top center head
%\rhead{\firstxmark} % Top right header
\lfoot{\lastxmark} % Bottom left footer
\cfoot{} % Bottom center footer
\rfoot{Page\ \thepage\ of\ \protect\pageref{LastPage}} % Bottom right footer
\renewcommand\headrulewidth{0.4pt} % Size of the header rule
\renewcommand\footrulewidth{0.4pt} % Size of the footer rule

\setlength\parindent{0pt} % Removes all indentation from paragraphs

%----------------------------------------------------------------------------------------
%	CODE INCLUSION CONFIGURATION
%----------------------------------------------------------------------------------------

\definecolor{MyDarkGreen}{rgb}{0.0,0.4,0.0} % This is the color used for comments
\lstloadlanguages{PHP,Java,Sql,HTML} % Load Perl syntax for listings, for a list of other languages supported see: ftp://ftp.tex.ac.uk/tex-archive/macros/latex/contrib/listings/listings.pdf

\lstset{ %
  backgroundcolor=\color{SpringGreen},   % choose the background color; you must add \usepackage{color} or \usepackage{xcolor}
  basicstyle=\footnotesize,        % the size of the fonts that are used for the code
  breakatwhitespace=false,         % sets if automatic breaks should only happen at whitespace
  breaklines=true,                 % sets automatic line breaking
  captionpos=b,                    % sets the caption-position to bottom
  commentstyle=\color{mygreen},    % comment style
  deletekeywords={...},            % if you want to delete keywords from the given language
  escapeinside={\%*}{*)},          % if you want to add LaTeX within your code
  extendedchars=true,              % lets you use non-ASCII characters; for 8-bits encodings only, does not work with UTF-8
  frame=single,                    % adds a frame around the code
  keywordstyle=\color{blue},       % keyword style
  language=PHP,                 % the language of the code
  morekeywords={*,...},            % if you want to add more keywords to the set
  numbers=left,                    % where to put the line-numbers; possible values are (none, left, right)
  numbersep=5pt,                   % how far the line-numbers are from the code
  stepnumber=3,    
  firstnumber=1,
  numberfirstline=false
  %numberstyle=\tiny\color{gray}, % the style that is used for the line-numbers
  rulecolor=\color{black},         % if not set, the frame-color may be changed on line-breaks within not-black text (e.g. comments (green here))
  showspaces=false,                % show spaces everywhere adding particular underscores; it overrides 'showstringspaces'
  showstringspaces=false,          % underline spaces within strings only
  showtabs=false,                  % show tabs within strings adding particular underscores
  %stepnumber=2,                    % the step between two line-numbers. If it's 1, each line will be numbered
  stringstyle=\color{mymauve},     % string literal style
  tabsize=2,                       % sets default tabsize to 2 spaces
  title=\lstname                   % show the filename of files included with \lstinputlisting; also try caption instead of title
}


% Creates a new command to include a perl script, the first parameter is the filename of the script (without .pl), the second parameter is the caption
\newcommand{\phpscript}[1]{
\begin{itemize}
%\item[]\lstinputlisting[caption=#2,label=#2,firstline=#3,lastline=#4]{#1.php}
\item[]\lstinputlisting{/var/www/sm/#1.php}
\end{itemize}
}

%----------------------------------------------------------------------------------------
%	DOCUMENT STRUCTURE COMMANDS
%	Skip this unless you know what you're doing
%----------------------------------------------------------------------------------------

% Header and footer for when a page split occurs within a problem environment
\newcommand{\enterProblemHeader}[1]{
\nobreak\extramarks{#1}{#1 continued on next page\ldots}\nobreak
\nobreak\extramarks{#1 (continued)}{#1 continued on next page\ldots}\nobreak
}

% Header and footer for when a page split occurs between problem environments
\newcommand{\exitProblemHeader}[1]{
\nobreak\extramarks{#1 (continued)}{#1 continued on next page\ldots}\nobreak
\nobreak\extramarks{#1}{}\nobreak
}

\setcounter{secnumdepth}{0} % Removes default section numbers
\newcounter{homeworkProblemCounter} % Creates a counter to keep track of the number of problems

\newcommand{\homeworkProblemName}{}
\newenvironment{homeworkProblem}[1]{ % Makes a new environment called homeworkProblem which takes 1 argument (custom name) but the default is "Problem #"
%[Problem \arabic{homeworkProblemCounter}]
%\stepcounter{homeworkProblemCounter} % Increase counter for number of problems
\renewcommand{\homeworkProblemName}{#1} % Assign \homeworkProblemName the name of the problem
\begin{center}
\section{#1} % Make a section in the document with the custom problem count
\end{center}
\enterProblemHeader{#1} % Header and footer within the environment
%\exitProblemHeader{\homeworkProblemName}
}{
\exitProblemHeader{\homeworkProblemName} % Header and footer after the environment
}

\newcommand{\problemAnswer}[1]{ % Defines the problem answer command with the content as the only argument
\noindent\framebox[\columnwidth][c]{\begin{minipage}{0.98\columnwidth}#1\end{minipage}} % Makes the box around the problem answer and puts the content inside
}

\newcommand{\homeworkSectionName}{}
\newenvironment{homeworkSection}[1]{ % New environment for sections within homework problems, takes 1 argument - the name of the section
\renewcommand{\homeworkSectionName}{#1} % Assign \homeworkSectionName to the name of the section from the environment argument
\subsection{\homeworkSectionName} % Make a subsection with the custom name of the subsection
\enterProblemHeader{"asdasd"\ [\homeworkSectionName]}
%\enterProblemHeader{\homeworkProblemName\ [\homeworkSectionName]} % Header and footer within the environment
}
%{
%\enterProblemHeader{\homeworkProblemName} % Header and footer after the environment
%}

%----------------------------------------------------------------------------------------
%	NAME AND CLASS SECTION
%----------------------------------------------------------------------------------------

\newcommand{\hmwkTitle}{Assignment\ \#8} % Assignment title
\newcommand{\hmwkSubTitle}{Security Test Cases} %Subtitle
\newcommand{\hmwkDueDate}{Friday,\ April\ 19,\ 2013} % Due date
\newcommand{\hmwkClass}{Security\ Testing} % Course/class
%\newcommand{\hmwkClassTime}{13:30am} % Class/lecture time
%\newcommand{\hmwkClassInstructor}{Jones} % Teacher/lecturer
\newcommand{\hmwkAuthorName}{Fabrizio\ Zeni} % Your name
\newcommand{\hmwkAuthorSId}{Student Id: 153465} % Student Id

%----------------------------------------------------------------------------------------
%	TITLE PAGE
%----------------------------------------------------------------------------------------

\title{
\vspace{2in}
\textmd{\textbf{\hmwkClass:\ \hmwkTitle}}\\
\textmd{\normalsize{\textbf{\hmwkSubTitle}}}\\
%\normalsize\vspace{0.1in}\small{Due\ on\ \hmwkDueDate}\\
%\vspace{0.1in}\large{\textit{\hmwkClassInstructor\ \hmwkClassTime}}
\vspace{3in}
}

\author{\textbf{\hmwkAuthorName} \\ \small{\hmwkAuthorSId}}
\date{} % Insert date here if you want it to appear below your name

%----------------------------------------------------------------------------------------

\begin{document}

\maketitle

%----------------------------------------------------------------------------------------
%	TABLE OF CONTENTS
%----------------------------------------------------------------------------------------

%\setcounter{tocdepth}{1} % Uncomment this line if you don't want subsections listed in the ToC

\newpage
\tableofcontents
\newpage

%----------------------------------------------------------------------------------------
%	VULNERABILITY 11
%----------------------------------------------------------------------------------------

\begin{homeworkProblem}{Vulnerability 11}
\label{sec:V11}
\subsection{Brief Analysis}
\begin{center}
	File: AddAssignment.php
	\begin{table}[h]
		\begin{center}
    		\begin{tabular}{ | c | c |}
    			\hline
    			\textbf{VARIABLE} & \textbf{RESULT} \\ \hline
  				page & true \\ \hline
  				page2 & true \\ \hline
  				selectclass & true \\ \hline
    		\end{tabular}
    	\end{center}
   \end{table}
\end{center}
\subsection{JWebUnit test cases}
\subsubsection{prepare and cleanup}
\begin{lstlisting}[language=Java,caption=prepare function]
	public void prepare(){
        tester = new WebTester();
        tester.setBaseUrl("http://localhost/sm/");
        tester.beginAt("index.php");
        Functions.login(tester,"teacher");
        Functions.click(tester,"Music",0);
        tester.assertMatch("Class Settings");
        Functions.click(tester,"Assignments",0);
        tester.assertMatch("Manage Assignments");
	}
\end{lstlisting}
\begin{lstlisting}[language=Java,caption=cleanup function]
	public void cleanup(){
		Functions.click(tester,"Log Out",0);
		tester = null;
	}
\end{lstlisting}
In these two functions there is nothing special, just navigation and call to the login/logout utilities.
\begin{center}
	\textit{\small Continues on the next page ...}
\end{center}
\clearpage
\subsubsection{page}
%\lstinputlisting[firstline=28,lastline=32,language=Java,caption=jwebunit test code for \textit{page}]{/home/ashen/workspace/TestingFramework/src/tests/V11.java}
\begin{lstlisting}[language=Java,caption=jwebunit test code for \textit{page}]
	public void page(){
		Vulnerabilities.page(tester,"assignments","Add");
        tester.assertMatch("Add New Assignment");
        tester.assertLinkNotPresentWithText("malicious");
	}
\end{lstlisting}
%\lstinputlisting[firstline=7,lastline=13,language=Java,caption=function for the \textit{page} vulnerability]{/home/ashen/workspace/TestingFramework/src/util/Vulnerabilities.java}
\begin{lstlisting}[language=Java,caption=function for the \textit{page} vulnerability]
	public static void page(WebTester tester,String formName,String buttonName){
        IElement page = tester.getElementByXPath("//form[@name='" + formName + "']//input[@name='page']");
        String oldValue = page.getAttribute("value");
        page.setAttribute("value",oldValue +"'><a href='http://www.unitn.it'>malicious</a><br'");
        if(buttonName!=null)
        	Functions.click(tester,buttonName,1);
	}
\end{lstlisting}
This code does the test for \textit{page}. In order to catch the correct hidden field it was necessary to filter the form first, because there were two hidden fields with the same name and the first is not the one triggered by the buttons.
So the function retrieves the page2 input element and stores it into the \emph{oldValue} variable, which at line 6 is concatenated to the malicious link and inserted into the page value.
\subsubsection{page2}
%\lstinputlisting[firstline=35,lastline=39,language=Java,caption=jwebunit test code for \textit{page2}]{/home/ashen/workspace/TestingFramework/src/tests/V11.java}
\begin{lstlisting}[language=Java,caption=jwebunit test code for \textit{page2}]
	public void page2(){
		Vulnerabilities.page2(tester,"assignments","Add");
        tester.assertMatch("Add New Assignment");
        tester.assertLinkNotPresentWithText("malicious");
	}
\end{lstlisting}
%\lstinputlisting[firstline=15,lastline=23,language=Java,caption=function for the \textit{page2} vulnerability]{/home/ashen/workspace/TestingFramework/src/util/Vulnerabilities.java}
\begin{lstlisting}[language=Java,caption=function for the \textit{page2} vulnerability]
	public static void page2(WebTester tester,String formName,String buttonName){
		IElement page2 = tester.getElementByXPath("//form[@name='" + formName + "']//input[@name='page2']");
		IElement button = tester.getElementByXPath("//input [@value='" + buttonName + "']");
		String onClick = button.getAttribute("onClick");
		String[] fixedValues = Functions.page2Fix(formName, onClick);
		fixedValues[0] = fixedValues[0].replace("'","");
		page2.setAttribute("value",fixedValues[0] + "'><a href='http://www.unitn.it'>malicious</a><br'");
		button.setAttribute("onClick",fixedValues[1]);
		Functions.click(tester,buttonName,1);
	}
\end{lstlisting}
The page2 vulnerability was more subtle to automatically trigger. That was due to the fact that the form buttons have a \textit{javascript} code in the attribute \textbf{onClick}, which write on the page2 value. So that in order to prevent the button from modify the injected value, at line 3 the button element is retrieved, then we get the value of the onClick attribute, which is processed by the \emph{page2Fix function} -  which purge the attribute from any command that modifies the page2 value and returns the value for page2 and the other instructions that need to be put back into the attribute.
\subsubsection{selectclass}
%\lstinputlisting[firstline=42,lastline=46,language=Java,caption=jwebunit test code for \textit{selectclass}]{/home/ashen/workspace/TestingFramework/src/tests/V11.java}
%\lstinputlisting[firstline=34,lastline=39,language=Java,caption=function for the \textit{selectclass} vulnerability]{/home/ashen/workspace/TestingFramework/src/util/Vulnerabilities.java}
\begin{lstlisting}[language=Java,caption=jwebunit test code for \textit{selectclass}]
	public void selectclass(){
		Vulnerabilities.selectclass(tester,"assignments","Add");
        tester.assertMatch("Add New Assignment");
        tester.assertLinkNotPresentWithText("malicious");
	}
\end{lstlisting}
\begin{lstlisting}[language=Java,caption=function for the \textit{selectclass} vulnerability]
	public static void selectclass(WebTester tester,String formName,String buttonName){
        IElement selectclass = tester.getElementByXPath("//form[@name='" + formName + "']//input[@name='selectclass']");
        String oldValue = selectclass.getAttribute("value");
        selectclass.setAttribute("value",oldValue+"'><a href='http://www.unitn.it'>malicious</a><br'");
        Functions.click(tester,buttonName,1);
	}
\end{lstlisting}
The selectclass vulnerability was almost straightforward and differs from the \textit{page} function just in the attribute name in the XPath expression. 
\end{homeworkProblem}
\clearpage

%----------------------------------------------------------------------------------------
%	VULNERABILITY 13
%----------------------------------------------------------------------------------------

\begin{homeworkProblem}{Vulnerability 13}
\label{sec:V13}
\subsection{Brief Analysis}
\begin{center}
	File: AddAttendance.php
	\begin{table}[h]
		\begin{center}
    		\begin{tabular}{ | c | c | }
    			\hline
    			\textbf{VARIABLE} & \textbf{RESULT} \\ \hline
  				page & true \\ \hline
  				page2 & true \\ \hline
  				student & true \\ \hline
  				semester & true \\ \hline
    		\end{tabular}
    	\end{center}
   \end{table}
\end{center}
\subsection{JWebUnit test cases}
\subsubsection{prepare and cleanup}
\begin{lstlisting}[language=Java,caption=prepare function]
	public void prepare(){
        tester = new WebTester();
        tester.setBaseUrl("http://localhost/sm/");
        tester.beginAt("index.php");
        Functions.login(tester,"admin");
		Functions.click(tester,"Attendance",0);
		tester.assertMatch("Tardy");
	}
\end{lstlisting}
\begin{lstlisting}[language=Java,caption=cleanup function]
	public void cleanup(){
		Functions.click(tester,"Log Out",0);
		tester = null;
	}
\end{lstlisting}
\subsubsection{page and page2}
The code is adapted from the one of \textit{Vulnerability 11} at page ~\pageref{sec:V11}
\subsubsection{student}
\begin{lstlisting}[language=Java,caption=jwebunit test code for \textit{student}]
	public void student(){
		Vulnerabilities.selectInputVulnerability(tester,"registration","Add","student");
        tester.assertMatch("Add New Attendance Record");
        tester.assertLinkNotPresentWithText("malicious");
	}
\end{lstlisting}
\begin{lstlisting}[language=Java,caption=function for vulnerabilities over select input elements]
	public static void selectInputVulnerability(WebTester tester,String formName,String buttonName,String vulnerability){
		IElement selectInput = tester.getElementByXPath("//form[@name='" + formName +
				"']//select[@name='" + vulnerability + "']//option[@selected]");
        String oldValue = selectInput.getAttribute("value");
        selectInput.setAttribute("value",oldValue+"'><a href='http://www.unitn.it'>malicious</a><br /'");
        Functions.click(tester,buttonName,1);
	}
\end{lstlisting}
%\lstinputlisting[firstline=40,lastline=44,language=Java,caption=jwebunit test code for \textit{student}]{/home/ashen/workspace/TestingFramework/src/tests/V13.java}
%\lstinputlisting[firstline=51,lastline=57,language=Java,caption=function for vulnerabilities over select input elements]{/home/ashen/workspace/TestingFramework/src/util/Vulnerabilities.java}
In this case the input element was a \textbf{select}, so the XPATH expression was modified with \emph{//option[@selected]} to catch the selected option. The remaining part of the code is almost equivalent to the \textit{page} one.
\subsubsection{semester}
\begin{lstlisting}[language=Java,caption=jwebunit test code for \textit{semester}]
	public void semester(){
		Vulnerabilities.selectInputVulnerability(tester,"registration","Add","semester");
        tester.assertMatch("Add New Attendance Record");
        tester.assertLinkNotPresentWithText("malicious");
	}
\end{lstlisting}
%\lstinputlisting[firstline=47,lastline=51,language=Java,caption=jwebunit test code for \textit{semester}]{/home/ashen/workspace/TestingFramework/src/tests/V13.java}
The semester test is a copy-paste of the student one.
\end{homeworkProblem}
\clearpage

%----------------------------------------------------------------------------------------
%	VULNERABILITY 16
%----------------------------------------------------------------------------------------

\begin{homeworkProblem}{Vulnerability 16}
\subsection{Brief Analysis}
\begin{center}
	File: AddAnnouncements.php
	\begin{table}[h]
		\begin{center}
    		\begin{tabular}{ | c | c | c |}
    			\hline
    			\textbf{VARIABLE} & \textbf{RESULT} \\ \hline
  				page & true \\ \hline
  				page2 & true \\ \hline
    		\end{tabular}
    	\end{center}
   \end{table}
\end{center}
\subsection{JWebUnit test cases}
The code is adapted from the one of \textit{Vulnerability 11} at page ~\pageref{sec:V11}
\end{homeworkProblem}
\clearpage

%----------------------------------------------------------------------------------------
%	VULNERABILITY 18
%----------------------------------------------------------------------------------------

\begin{homeworkProblem}{Vulnerability 18}
\subsection{Brief Analysis}
\begin{center}
	File: AddUser.php
	\begin{table}[h]
		\begin{center}
    		\begin{tabular}{ | c | c | }
    			\hline
    			\textbf{VARIABLE} & \textbf{RESULT} \\ \hline
  				page & true \\ \hline
  				page2 & true \\ \hline
    		\end{tabular}
    	\end{center}
   \end{table}
\end{center}
\subsection{JWebUnit test cases}
The code is adapted from the one of \textit{Vulnerability 11} at page ~\pageref{sec:V11}
\end{homeworkProblem}
\clearpage

%----------------------------------------------------------------------------------------
%	VULNERABILITY 19
%----------------------------------------------------------------------------------------

\begin{homeworkProblem}{Vulnerability 19}
\subsection{Brief Analysis}
\begin{center}
	File: AddTerm.php
	\begin{table}[h]
		\begin{center}
    		\begin{tabular}{ | c | c | }
    			\hline
    			\textbf{VARIABLE} & \textbf{RESULT} \\ \hline
  				page & true \\ \hline
  				page2 & true \\ \hline
    		\end{tabular}
    	\end{center}
   \end{table}
\end{center}
\subsection{JWebUnit test cases}
The code is adapted from the one of \textit{Vulnerability 11} at page ~\pageref{sec:V11}
\end{homeworkProblem}
\clearpage

%----------------------------------------------------------------------------------------
%	VULNERABILITY 30,31
%----------------------------------------------------------------------------------------

\begin{homeworkProblem}{Vulnerability 30,31}
\subsection{Brief Analysis}
\label{sec:V30}
\begin{center}
	File: ViewAssignments.php
	\begin{table}[h]
		\begin{center}
    		\begin{tabular}{ | c | c | c |}
    			\hline
    			\textbf{VARIABLE} & \textbf{RESULT} \\ \hline
  				page & true \\ \hline
  				page2 & true \\ \hline
				coursename & true \\ \hline
				assignment[5] & true \\ \hline
    		\end{tabular}
    	\end{center}
   \end{table}
\end{center}
\subsection{JWebUnit test cases}
\subsubsection{prepare and cleanup}
\begin{lstlisting}[language=Java,caption=prepare function]
	public void prepare(){
        tester = new WebTester();
        tester.setBaseUrl("http://localhost/sm/");
        tester.beginAt("index.php");
        Functions.login(tester,"student");
        Functions.click(tester,"Music",0);
        tester.assertMatch("Class Settings");
	}
\end{lstlisting}
\begin{lstlisting}[language=Java,caption=cleanup function]
	public void cleanup() {
		Functions.click(tester, "Log Out", 0);
		// BEGIN COURSENAME CLEANUP
		Functions.login(tester, "admin");
		Functions.click(tester, "Classes", 0);
		tester.assertMatch("Manage Classes");
		IElement myCheckbox = tester
				.getElementByXPath("//td[text()='Music']/..//input[@type='checkbox']");
		tester.setWorkingForm("classes");
		tester.checkCheckbox("delete[]", myCheckbox.getAttribute("value"));
		Functions.click(tester, "Edit", 1);
		tester.assertMatch("Edit Class");
		tester.setTextField("title","Music");
		Functions.click(tester,"Edit Class", 1);
		Functions.click(tester, "Log Out", 0);
		// END COURSENAME CLEANUP
		tester = null;
	}
\end{lstlisting}
\clearpage
\subsubsection{page}
\begin{lstlisting}[language=Java,caption=jwebunit test code for \textit{page}]
	public void page(){
		Vulnerabilities.page(tester,"student",null);
	    Functions.click(tester,"Assignments",0);
	    tester.assertMatch("View Assignments");
	    tester.assertMatch("verifica di prova");
        tester.assertLinkNotPresentWithText("malicious");
	}
\end{lstlisting}
\subsubsection{page2}
\begin{lstlisting}[language=Java,caption=jwebunit test code for \textit{page2}]
	public void page2(){
		Vulnerabilities.page2Link(tester,"student","Assignments","document.student.submit();");
	    tester.assertMatch("View Assignments");
	    tester.assertMatch("verifica di prova");
        tester.assertLinkNotPresentWithText("malicious");
	}
\end{lstlisting}
\begin{lstlisting}[language=Java,caption=function for the page2 vulnerability with links]
	public static void page2Link(WebTester tester,String formName,String linkName,String hrefValue){
		IElement page2 = tester.getElementByXPath("//form[@name='" + formName + "']//input[@name='page2']");
		IElement link = tester.getElementByXPath("//a[text()='" + linkName + "']");
		link.setAttribute("href","javascript: " + hrefValue);
		Integer page2Value = Functions.getPage2(linkName);
		page2.setAttribute("value",page2Value + "'><a href='http://www.unitn.it'>malicious</a><br'");
		Functions.click(tester,linkName,0);
	}
\end{lstlisting}
Here a modified version of the page2 utility function is used. That is due to the fact that in this case we have to modify a link instead of a button.
\begin{center}
	\textit{\small Continues on the next page ...}
\end{center}
\clearpage
\subsubsection{coursename}
\begin{lstlisting}[language=Java,caption=jwebunit test code for \textit{coursename}]
	public void coursename() {
		Functions.click(tester, "Log Out", 0);
		tester.assertMatch("TutttoBBBene");
		// INJECTING A LINK IN THE COURSENAME
		Functions.login(tester, "admin");
		Functions.click(tester, "Classes", 0);
		tester.assertMatch("Manage Classes");
		IElement myCheckbox = tester
				.getElementByXPath("//td[text()='Music']/..//input[@type='checkbox']");
		tester.setWorkingForm("classes");
		tester.checkCheckbox("delete[]", myCheckbox.getAttribute("value"));
		Functions.click(tester, "Edit", 1);
		tester.assertMatch("Music");
		tester.assertMatch("Edit Class");
		Vulnerabilities.textFieldVulnerability(tester, "editclass", "title",
				"Edit Class");
		tester.assertLinkPresentWithText("a");
		Functions.click(tester, "Log Out", 0);
		// CHECKING THE VULNERABILITY
		Functions.login(tester, "student");
		Functions.click(tester, "Music", 0);
		tester.assertMatch("Class Settings");
		Functions.click(tester, "Assignments", 0);
		tester.assertMatch("View Assignments");
		tester.assertLinkNotPresentWithText("a");
	}
\end{lstlisting}
This test is a bit more verbose, because in order to test the \textit{coursename} vulnerability a injection made through an admin account is required.
\begin{lstlisting}[language=Java,caption=function used to inject links in textfields]
	public static void textFieldVulnerability(WebTester tester,
			String formName, String fieldName,String buttonName) {
		String oldValue = tester.getElementByXPath("//input [@name='" + fieldName + "']").getAttribute("value");
		tester.setTextField(fieldName,oldValue + "<a href>a</a>");
		Functions.click(tester,buttonName, 1);
	}
\end{lstlisting}
For this vulnerability, I wrote a generic function in the Vulnerability class which is able to process vulnerabilities over text fields.
\end{homeworkProblem}
\clearpage

%----------------------------------------------------------------------------------------
%	VULNERABILITY 37
%----------------------------------------------------------------------------------------

\begin{homeworkProblem}{Vulnerability 37}
\label{sec:V37}
\subsection{Brief Analysis}
\begin{center}
	File: EditAssignment.php
	\begin{table}[h]
		\begin{center}
    		\begin{tabular}{ | c | c |}
    			\hline
    			\textbf{VARIABLE} & \textbf{RESULT} \\ \hline
  				page & true \\ \hline
  				page2 & true \\ \hline
  				selectclass & true \\ \hline
  				delete & true \\ \hline
    		\end{tabular}
    	\end{center}
   \end{table}
\end{center}
\subsection{JWebUnit test cases}
\subsubsection{prepare and cleanup}
%\lstinputlisting[firstline=17,lastline=30,language=Java,caption=prepare function]{/home/ashen/workspace/TestingFramework/src/tests/V37.java}
\begin{lstlisting}[language=Java,caption=prepare function]
	public void prepare(){
		tester = new WebTester();
		tester.setBaseUrl("http://localhost/sm/");
		tester.beginAt("index.php");
		Functions.login(tester,"teacher");
		Functions.click(tester,"Music",0);
		tester.assertMatch("Class Settings");
		Functions.click(tester,"Assignments",0);
		tester.assertMatch("Manage Assignments");
		tester.assertMatch("verifica di prova");
		IElement myCheckbox = tester.getElementByXPath("//td[text()='prova2']/..//input[@type='checkbox']");
		tester.setWorkingForm("assignments");
		tester.checkCheckbox("delete[]",myCheckbox.getAttribute("value"));
	}
\end{lstlisting}
The prepare functions was a bit longer this time, because in order to access to the reported page one of the assignment has to be checked in the checkbox element. This is done by retrieving the line of the assignment \emph{prova} and finally we set insert in the \textit{delete[]} the value of the selected assignment.
\begin{lstlisting}[language=Java,caption=cleanup function]
	public void cleanup(){
		Functions.click(tester,"Log Out",0);
		tester = null;
	}
\end{lstlisting}
\subsubsection{page, page2 and selectclass}
The code is adapted from the one of \textit{Vulnerability 11} at page ~\pageref{sec:V11}
\subsubsection{delete}
\label{sec:deleteV37}
\begin{lstlisting}[language=Java,caption=jwebunit test code for \textit{delete}]
	public void delete(){
		Vulnerabilities.delete(tester,"assignments","Edit","prova2");
		tester.assertMatch("EditAssignment.php: Unable to retrieve");
		tester.assertLinkNotPresentWithText("malicious");
	}
\end{lstlisting}
\begin{lstlisting}[language=Java,caption=function for the \textit{delete} vulnerability]
	public static void delete(WebTester tester,String formName,String buttonName,String checkBoxText){
		IElement myCheckBox = tester.getElementByXPath("//td[text()='" + checkBoxText
				+ "']/..//input[@type='checkbox']");
		String oldValue = myCheckBox.getAttribute("value");
		myCheckBox.setAttribute("value",oldValue + "';<a href=http://www.unitn.it>malicious</a>");
		tester.assertButtonPresentWithText("Edit");
		System.err.println(myCheckBox.getAttribute("value"));
		Functions.click(tester,buttonName,1);
	}
\end{lstlisting}
The interesting thing of this case is that even a \emph{sql injection} is possible by putting another query after the semicolon.
\end{homeworkProblem}
\clearpage

%----------------------------------------------------------------------------------------
%	VULNERABILITY 41
%----------------------------------------------------------------------------------------

\begin{homeworkProblem}{Vulnerability 41}
\subsection{Brief Analysis}
\begin{center}
	File: EditAnnouncement.php
	\begin{table}[h]
		\begin{center}
    		\begin{tabular}{ | c | c |}
    			\hline
    			\textbf{VARIABLE} & \textbf{RESULT} \\ \hline
  				page & true \\ \hline
  				page2 & true \\ \hline
  				delete & true \\ \hline
    		\end{tabular}
    	\end{center}
   \end{table}
\end{center}
\subsection{JWebUnit test cases}
The code is adapted from the one of \textit{Vulnerability 37} at page ~\pageref{sec:V37}
\end{homeworkProblem}
\clearpage

%----------------------------------------------------------------------------------------
%	VULNERABILITY 44
%----------------------------------------------------------------------------------------

\begin{homeworkProblem}{Vulnerability 44}
\subsection{Brief Analysis}
\begin{center}
	File: EditTerm.php
	\begin{table}[h]
		\begin{center}
    		\begin{tabular}{ | c | c |}
    			\hline
    			\textbf{VARIABLE} & \textbf{RESULT} \\ \hline
  				page & true \\ \hline
  				page2 & true \\ \hline
  				delete & true \\ \hline
    		\end{tabular}
    	\end{center}
   \end{table}
\end{center}
\subsection{JWebUnit test cases}
The code is adapted from the one of \textit{Vulnerability 37} at page ~\pageref{sec:V37}
\end{homeworkProblem}
\clearpage

%----------------------------------------------------------------------------------------
%	VULNERABILITY 54
%----------------------------------------------------------------------------------------

\begin{homeworkProblem}{Vulnerability 54}
\subsection{Brief Analysis}
\begin{center}
	File: Login.php
	\begin{table}[h]
		\begin{center}
    		\begin{tabular}{ | c | c |}
    			\hline
    			\textbf{VARIABLE} & \textbf{RESULT} \\ \hline
  				text & true \\ \hline
    		\end{tabular}
    	\end{center}
   \end{table}
\end{center}
\subsection{JWebUnit test cases}
\subsubsection{prepare and cleanup}
\begin{lstlisting}[language=Java,caption=prepare function]
	public void prepare(){
        tester = new WebTester();
        tester.setBaseUrl("http://localhost/sm/");
        tester.beginAt("index.php");
    	Functions.login(tester,"admin");
        Functions.click(tester,"School",0);
        tester.assertMatch("Manage School Information");
	}
\end{lstlisting}
\begin{lstlisting}[language=Java,caption=cleanup function]
	public void cleanup(){
        tester.assertMatch("Today's Message");
        Functions.login(tester, "admin");
        tester.clickLinkWithText("School");
        tester.assertMatch("Manage School Information");
        tester.setTextField("sitetext", oldValue);
        Functions.click(tester," Update ",1);
        Functions.click(tester,"Log Out",0);
        tester = null;
	}
\end{lstlisting}
\subsubsection{text}
\begin{lstlisting}[language=Java,caption=jwebunit test code for \textit{text}]
	public void siteText(){
        oldValue = tester.getElementByXPath("//textarea [@name='sitetext']").getTextContent();
        tester.setTextField("sitetext", "<a href=\"http://www.unitn.it\">malicious</a>");
        Functions.click(tester," Update ",1);
        Functions.click(tester,"Log Out",0);
        tester.assertLinkNotPresentWithText("malicious");
	}
\end{lstlisting}
\end{homeworkProblem}
\clearpage

%----------------------------------------------------------------------------------------
%	VULNERABILITY 63
%----------------------------------------------------------------------------------------

\begin{homeworkProblem}{Vulnerability 63}
\subsection{Brief Analysis}
\begin{center}
	File: AddTeacher.php
	\begin{table}[h]
		\begin{center}
    		\begin{tabular}{ | c | c | }
    			\hline
    			\textbf{VARIABLE} & \textbf{RESULT} \\ \hline
  				page & true \\ \hline
  				page2 & true \\ \hline
    		\end{tabular}
    	\end{center}
   \end{table}
\end{center}
\subsection{JWebUnit test cases}
The code is adapted from the one of \textit{Vulnerability 11} at page ~\pageref{sec:V11}
\end{homeworkProblem}
\clearpage

%----------------------------------------------------------------------------------------
%	VULNERABILITY 70
%----------------------------------------------------------------------------------------

\begin{homeworkProblem}{Vulnerability 70}
\subsection{Brief Analysis}
\begin{center}
	File: AddStudent.php
	\begin{table}[h]
		\begin{center}
    		\begin{tabular}{ | c | c | }
    			\hline
    			\textbf{VARIABLE} & \textbf{RESULT} \\ \hline
  				page & true \\ \hline
  				page2 & true \\ \hline
    		\end{tabular}
    	\end{center}
   \end{table}
\end{center}
\subsection{JWebUnit test cases}
The code is adapted from the one of \textit{Vulnerability 11} at page ~\pageref{sec:V11}
\end{homeworkProblem}
\clearpage

%----------------------------------------------------------------------------------------
%	VULNERABILITY 71
%----------------------------------------------------------------------------------------

\begin{homeworkProblem}{Vulnerability 71}
\subsection{Brief Analysis}
\begin{center}
	File: AddSemester.php
	\begin{table}[h]
		\begin{center}
    		\begin{tabular}{ | c | c | }
    			\hline
    			\textbf{VARIABLE} & \textbf{RESULT} \\ \hline
  				page & true \\ \hline
  				page2 & true \\ \hline
    		\end{tabular}
    	\end{center}
   \end{table}
\end{center}
\subsection{JWebUnit test cases}
The code is adapted from the one of \textit{Vulnerability 11} at page ~\pageref{sec:V11}
\end{homeworkProblem}
\clearpage

%----------------------------------------------------------------------------------------
%	VULNERABILITY 76
%----------------------------------------------------------------------------------------

\begin{homeworkProblem}{Vulnerability 76}
\subsection{Brief Analysis}
\begin{center}
	File: EditAnnouncement.php
	\begin{table}[h]
		\begin{center}
    		\begin{tabular}{ | c | c | }
    			\hline
    			\textbf{VARIABLE} & \textbf{RESULT} \\ \hline
  				page & true \\ \hline
  				page2 & true \\ \hline
  				selectclass & true \\ \hline
  				assignment & true \\ \hline
  				delete & true \\ \hline
    		\end{tabular}
    	\end{center}
   \end{table}
\end{center}
\subsection{JWebUnit test cases}
\subsubsection{prepare and cleanup}
\begin{lstlisting}[language=Java,caption=prepare function]
	public void prepare(){
		tester = new WebTester();
		tester.setBaseUrl("http://localhost/sm/");
		tester.beginAt("index.php");
		Functions.login(tester,"teacher");
		Functions.click(tester,"Music",0);
		tester.assertMatch("Class Settings");
		Functions.click(tester,"Grades",0);
		tester.assertMatch("Date Submitted");
		IElement myCheckbox = tester.getElementByXPath("//td[text()='Harry Potter']/..//input[@type='checkbox']");
		tester.setWorkingForm("grades");
		tester.checkCheckbox("delete[]",myCheckbox.getAttribute("value"));
	}
\end{lstlisting}
\begin{lstlisting}[language=Java,caption=cleanup function]
	public void cleanup(){
		Functions.click(tester,"Log Out",0);
		tester = null;
	}
\end{lstlisting}
\subsubsection{page,page2,selectclass and delete}
The code is adapted from the one of \textit{Vulnerability 37} at page ~\pageref{sec:V37}
\subsubsection{assignment}
\begin{lstlisting}[language=Java,caption=jwebunit test code for \textit{assignment}]
	public void assignment(){
		Vulnerabilities.selectInputVulnerability(tester,"grades","Edit","assignment");
		tester.assertMatch("EditGrade.php: Unable to retrieve");
		tester.assertLinkNotPresentWithText("malicious");
	}
\end{lstlisting}
\lstinputlisting[firstline=4,lastline=4,caption=EditGrade.php read of assignment]{/home/ashen/university/Security/sm/EditGrade.php}
In this case, the input element is a \textit{select}, but the posted variable is printed inside an sql query - so as already said for \emph{Vulnerability 37} - an Sql Injection is also possible.
\end{homeworkProblem}
\clearpage

%----------------------------------------------------------------------------------------
%	VULNERABILITY 85
%----------------------------------------------------------------------------------------

\begin{homeworkProblem}{Vulnerability 85}
\subsection{Brief Analysis}
\begin{center}
	File: EditSemester.php
	\begin{table}[h]
		\begin{center}
    		\begin{tabular}{ | c | c | }
    			\hline
    			\textbf{VARIABLE} & \textbf{RESULT} \\ \hline
  				page & true \\ \hline
  				page2 & true \\ \hline
  				delete & true \\ \hline
    		\end{tabular}
    	\end{center}
   \end{table}
\end{center}
\subsection{JWebUnit test cases}
The code is adapted from the one of \textit{Vulnerability 37} at page ~\pageref{sec:V37}
\end{homeworkProblem}
\clearpage

%----------------------------------------------------------------------------------------
%	VULNERABILITY 87
%----------------------------------------------------------------------------------------

\begin{homeworkProblem}{Vulnerability 87}
\subsection{Brief Analysis}
\begin{center}
	File: ViewClassSettings.php
	\begin{table}[h]
		\begin{center}
    		\begin{tabular}{ | c | c |}
    			\hline
    			\textbf{VARIABLE} & \textbf{RESULT} \\ \hline
  				page & true \\ \hline
  				page2 & true \\ \hline
  				selectclass & true \\ \hline
    		\end{tabular}
    	\end{center}
   \end{table}
\end{center}
\subsection{JWebUnit test cases}
The code is adapted from the one of \textit{Vulnerability 11} at page ~\pageref{sec:V11}
\end{homeworkProblem}

%----------------------------------------------------------------------------------------
%	VULNERABILITY 88,89
%----------------------------------------------------------------------------------------

\begin{homeworkProblem}{Vulnerability 88,89}
\subsection{Brief Analysis}
\begin{center}
	V88 File: ViewClassSettings.php
	V89 File: ClassSettings.php
	\begin{table}[h]
		\begin{center}
    		\begin{tabular}{ | c | c |}
    			\hline
    			\textbf{VARIABLE} & \textbf{RESULT} \\ \hline
  				page & true \\ \hline
  				page2 & true \\ \hline
  				selectclass & true \\ \hline
    		\end{tabular}
    	\end{center}
   \end{table}
\end{center}
The vulnerabilities 88 and 89 are almost the same of 87, with the difference that them are visible from student (V88) and teacher (V89) accounts instead of a parent one.
\end{homeworkProblem}
\clearpage

%----------------------------------------------------------------------------------------
%	VULNERABILITY 90
%----------------------------------------------------------------------------------------

\begin{homeworkProblem}{Vulnerability 90}
\subsection{Brief Analysis}
\begin{center}
	File: ParentViewStudents.php
	\begin{table}[h]
		\begin{center}
    		\begin{tabular}{ | c | c | }
    			\hline
    			\textbf{VARIABLE} & \textbf{RESULT} \\ \hline
  				page & true \\ \hline
  				page2 & true \\ \hline
    		\end{tabular}
    	\end{center}
   \end{table}
\end{center}
\subsection{JWebUnit test cases}
The code is adapted from the one of \textit{Vulnerability 11} at page ~\pageref{sec:V11} for the \textit{page} vulnerability, while using the modified version of \textit{Vulnerability 30} at page ~\pageref{sec:V30} for the \textit{page2} vulnerability.
\end{homeworkProblem}
\clearpage

%----------------------------------------------------------------------------------------
%	VULNERABILITY 92
%----------------------------------------------------------------------------------------

\begin{homeworkProblem}{Vulnerability 92}
\subsection{Brief Analysis}
\begin{center}
	File: ManageSchoolInfo.php
	\begin{table}[h]
		\begin{center}
    		\begin{tabular}{ | c | c | }
    			\hline
    			\textbf{VARIABLE} & \textbf{RESULT} \\ \hline
  				page & true \\ \hline
  				page2 & true \\ \hline
  				address & true \\ \hline
  				phone & true \\ \hline
    		\end{tabular}
    	\end{center}
   \end{table}
\end{center}
\subsection{JWebUnit test cases}
\subsubsection{prepare and cleanup}
\begin{lstlisting}[language=Java,caption=prepare function]
	public void prepare(){
        tester = new WebTester();
        tester.setBaseUrl("http://localhost/sm/");
        tester.beginAt("index.php");
        Functions.login(tester,"admin");
		tester.assertMatch("Manage Classes");
		Functions.click(tester, "School", 0);
        oldValue = tester.getElementByXPath("//input [@name='schooladdress']").getAttribute("value");
        Functions.click(tester, "Classes", 0);
        tester.assertMatch("Manage Classes");
	}
\end{lstlisting}
\begin{lstlisting}[language=Java,caption=cleanup function]
	public void cleanup(){
        tester.setTextField("schooladdress", oldValue);
        Functions.click(tester," Update ",1);
        tester.setTextField("schooladdress", oldValue);
		Functions.click(tester,"Log Out",0);
		tester = null;
	}
\end{lstlisting}
\subsubsection{page and page2}
The code is adapted from the one of \textit{Vulnerability 11} at page ~\pageref{sec:V11}
\subsubsection{address}
\begin{lstlisting}[language=Java,caption=jwebunit test code for \textit{address}]
	public void address(){
        Functions.login(tester,"admin");
        Functions.click(tester,"School",0);
        tester.assertMatch("Manage School Information");
        tester.setTextField("schooladdress", oldValue + "\'><a href>a</a>");
        Functions.click(tester," Update ",1);
        tester.assertLinkNotPresentWithText("a");
	}
\end{lstlisting}
\subsubsection{phone}
\begin{lstlisting}[language=Java,caption=jwebunit test code for \textit{phone}]

\end{lstlisting}
\end{homeworkProblem}
\clearpage

%----------------------------------------------------------------------------------------
%	VULNERABILITY 93
%----------------------------------------------------------------------------------------

\begin{homeworkProblem}{Vulnerability 93}
\subsection{Brief Analysis}
\begin{center}
	File: AddParent.php
	\begin{table}[h]
		\begin{center}
    		\begin{tabular}{ | c | c | }
    			\hline
    			\textbf{VARIABLE} & \textbf{RESULT} \\ \hline
  				page & true \\ \hline
  				page2 & true \\ \hline
    		\end{tabular}
    	\end{center}
   \end{table}
\end{center}
\subsection{JWebUnit test cases}
The code is adapted from the one of \textit{Vulnerability 11} at page ~\pageref{sec:V11}
\end{homeworkProblem}
\clearpage

%----------------------------------------------------------------------------------------
%	VULNERABILITY 105
%----------------------------------------------------------------------------------------

\begin{homeworkProblem}{Vulnerability 105}
\subsection{Brief Analysis}
\begin{center}
	File: Login.php
	\begin{table}[h]
		\begin{center}
    		\begin{tabular}{ | c | c | }
    			\hline
    			\textbf{VARIABLE} & \textbf{RESULT} \\ \hline
  				page & true \\ \hline
  				message & true \\ \hline
    		\end{tabular}
    	\end{center}
   \end{table}
\end{center}
\subsection{JWebUnit test cases}
\subsubsection{prepare and cleanup}
\begin{lstlisting}[language=Java,caption=prepare function]
		tester = new WebTester();
		tester.setBaseUrl("http://localhost/sm/");
		tester.beginAt("index.php");
		Functions.login(tester, "admin");
		Functions.click(tester, "School", 0);
		tester.assertMatch("Manage School Information");
		IElement textArea = tester.getElementByXPath("//textarea [@name='sitemessage']");
		oldValue = textArea.getTextContent();
		tester.setTextField("sitemessage", "<a href>malicious</a>");
		Functions.click(tester," Update ", 1);
		Functions.click(tester, "Log Out", 0);
		tester.assertMatch("Today's Message");
\end{lstlisting}
\begin{lstlisting}[language=Java,caption=cleanup function]
		Functions.login(tester, "admin");
		Functions.click(tester, "School", 0);
		tester.assertMatch("Manage School Information");
		tester.setTextField("sitemessage", oldValue);
		Functions.click(tester," Update ", 1);
		Functions.click(tester, "Log Out", 0);
		tester.assertLinkNotPresentWithText("malicious");
		tester = null;
\end{lstlisting}
\subsubsection{page}
\begin{lstlisting}[language=Java,caption=jwebunit test code for \textit{page}]
	public void page() {
		Vulnerabilities.page(tester, "login","Login");
		tester.assertMatch("Today's Message");
		tester.assertLinkNotPresentWithText("malicious");
	}
\end{lstlisting}
\subsubsection{message}
\begin{lstlisting}[language=Java,caption=jwebunit test code for \textit{message}]
		tester.assertLinkNotPresentWithText("malicious");
\end{lstlisting}
\end{homeworkProblem}
\clearpage

%----------------------------------------------------------------------------------------
%	VULNERABILITY 111
%----------------------------------------------------------------------------------------

\begin{homeworkProblem}{Vulnerability 111}
\subsection{Brief Analysis}
\begin{center}
	File: EditTeacher.php
	\begin{table}[h]
		\begin{center}
    		\begin{tabular}{ | c | c | }
    			\hline
    			\textbf{VARIABLE} & \textbf{RESULT} \\ \hline
  				page & true \\ \hline
  				page2 & true \\ \hline
  				delete & true \\ \hline
    		\end{tabular}
    	\end{center}
   \end{table}
\end{center}
\subsection{JWebUnit test cases}
The code is adapted from the one of \textit{Vulnerability 37} at page ~\pageref{sec:V37}
\subsection{Fix}
\end{homeworkProblem}
\clearpage

%----------------------------------------------------------------------------------------
%	VULNERABILITY 115
%----------------------------------------------------------------------------------------

\begin{homeworkProblem}{Vulnerability 115}
\subsection{Brief Analysis}
\begin{center}
	File: EditStudent.php
	\begin{table}[h]
		\begin{center}
    		\begin{tabular}{ | c | c | }
    			\hline
    			\textbf{VARIABLE} & \textbf{RESULT} \\ \hline
  				page & true \\ \hline
  				page2 & true \\ \hline
  				delete & true \\ \hline
    		\end{tabular}
    	\end{center}
   \end{table}
\end{center}
\subsection{JWebUnit test cases}
The code is adapted from the one of \textit{Vulnerability 37} at page ~\pageref{sec:V37}
\subsection{Fix}
\end{homeworkProblem}
\clearpage

%----------------------------------------------------------------------------------------
%	VULNERABILITY 126
%----------------------------------------------------------------------------------------

\begin{homeworkProblem}{Vulnerability 126}
\subsection{Brief Analysis}
\begin{center}
	File: ViewCourses.php
	\begin{table}[h]
		\begin{center}
    		\begin{tabular}{ | c | c | }
    			\hline
    			\textbf{VARIABLE} & \textbf{RESULT} \\ \hline
  				page & true \\ \hline
  				page2 & true \\ \hline
    		\end{tabular}
    	\end{center}
   \end{table}
\end{center}
\subsection{JWebUnit test cases}
The code is adapted from the one of \textit{Vulnerability 11} at page ~\pageref{sec:V11} for the \textit{page} vulnerability, while using the modified version of \textit{Vulnerability 30} at page ~\pageref{sec:V30} for the \textit{page2} vulnerability.
\end{homeworkProblem}
\clearpage

%----------------------------------------------------------------------------------------
%	VULNERABILITY 138
%----------------------------------------------------------------------------------------

\begin{homeworkProblem}{Vulnerability 138}
\subsection{Brief Analysis}
\begin{center}
	File: StudentViewCourses.php
	\begin{table}[h]
		\begin{center}
    		\begin{tabular}{ | c | c | }
    			\hline
    			\textbf{VARIABLE} & \textbf{RESULT} \\ \hline
  				page & true \\ \hline
  				page2 & true \\ \hline
    		\end{tabular}
    	\end{center}
   \end{table}
\end{center}
\subsection{JWebUnit test cases}
The code is adapted from the one of \textit{Vulnerability 11} at page ~\pageref{sec:V11} for the \textit{page} vulnerability, while using the modified version of \textit{Vulnerability 30} at page ~\pageref{sec:V30} for the \textit{page2} vulnerability.
\end{homeworkProblem}
\clearpage

%----------------------------------------------------------------------------------------
%	VULNERABILITY 141
%----------------------------------------------------------------------------------------

\begin{homeworkProblem}{Vulnerability 141}
\subsection{Brief Analysis}
\begin{center}
	File: AddClass.php
	\begin{table}[h]
		\begin{center}
    		\begin{tabular}{ | c | c | }
    			\hline
    			\textbf{VARIABLE} & \textbf{RESULT} \\ \hline
  				page & true \\ \hline
  				page2 & true \\ \hline
    		\end{tabular}
    	\end{center}
   \end{table}
\end{center}
\subsection{JWebUnit test cases}
The code is adapted from the one of \textit{Vulnerability 11} at page ~\pageref{sec:V11}.
\end{homeworkProblem}
\clearpage

%----------------------------------------------------------------------------------------
%	VULNERABILITY 142
%----------------------------------------------------------------------------------------

\begin{homeworkProblem}{Vulnerability 142}
\subsection{Brief Analysis}
\begin{center}
	File: ParentViewCourses.php
	\begin{table}[h]
		\begin{center}
    		\begin{tabular}{ | c | c | }
    			\hline
    			\textbf{VARIABLE} & \textbf{RESULT} \\ \hline
  				page & true \\ \hline
  				page2 & true \\ \hline
  				student & true \\ \hline
    		\end{tabular}
    	\end{center}
   \end{table}
\end{center}
\subsection{JWebUnit test cases}
\subsubsection{page and page2}
The code is adapted from the one of \textit{Vulnerability 11} at page ~\pageref{sec:V11}.
\subsubsection{student}
\begin{lstlisting}[language=Java,caption=jwebunit test code for \textit{student}]
	public void student(){
        IElement student = tester.getElementByXPath("//form[@name='student']//input[@name='student']");
        String oldValue = student.getAttribute("value");
        student.setAttribute("value",oldValue +"';<a href=http://www.unitn.it>malicious</a>");
		Functions.click(tester,"Classes",0);
        tester.assertMatch("ParentViewCourses.php: Unable to get the studentid 2");
        tester.assertLinkNotPresentWithText("malicious");
	}
\end{lstlisting}
\end{homeworkProblem}
\clearpage

%----------------------------------------------------------------------------------------
%	VULNERABILITY 146
%----------------------------------------------------------------------------------------

\begin{homeworkProblem}{Vulnerability 146}
\label{V146}
\subsection{Brief Analysis}
\begin{center}
	File: ViewAnnouncements.php
	\begin{table}[h]
		\begin{center}
    		\begin{tabular}{ | c | c | }
    			\hline
    			\textbf{VARIABLE} & \textbf{RESULT} \\ \hline
  				page & true \\ \hline
  				page2 & true \\ \hline
  				onpage & true \\ \hline
    		\end{tabular}
    	\end{center}
   \end{table}
\end{center}
\subsection{JWebUnit test cases}
\subsubsection{page and page2}
The code is adapted from the one of \textit{Vulnerability 11} at page ~\pageref{sec:V11}.
\subsubsection{onpage}
\textbf{DA VERIFICARE}
\begin{lstlisting}[language=Java,caption=jwebunit test code for \textit{onpage}]

\end{lstlisting}
\end{homeworkProblem}
\clearpage

%----------------------------------------------------------------------------------------
%	VULNERABILITY 147
%----------------------------------------------------------------------------------------

\begin{homeworkProblem}{Vulnerability 147}
\subsection{Brief Analysis}
\begin{center}
	File: ViewAnnouncements.php
	\begin{table}[h]
		\begin{center}
    		\begin{tabular}{ | c | c | }
    			\hline
    			\textbf{VARIABLE} & \textbf{RESULT} \\ \hline
  				page & true \\ \hline
  				page2 & true \\ \hline
  				onpage & true \\ \hline
    		\end{tabular}
    	\end{center}
   \end{table}
\end{center}
\subsection{JWebUnit test cases}
Is the same of \textit{Vulnerability 146} at page ~\pageref{V146}, but the exploit of the vulnerability is visibile through a \textit{student} account instead of a \textit{parent} one as in the former vulnerability.
\end{homeworkProblem}
\clearpage

%----------------------------------------------------------------------------------------
%	VULNERABILITY 148
%----------------------------------------------------------------------------------------

\begin{homeworkProblem}{Vulnerability 148}
\subsection{Brief Analysis}
\begin{center}
	File: ViewAnnouncements.php
	\begin{table}[h]
		\begin{center}
    		\begin{tabular}{ | c | c | }
    			\hline
    			\textbf{VARIABLE} & \textbf{RESULT} \\ \hline
  				page & true \\ \hline
  				page2 & true \\ \hline
  				onpage & true \\ \hline
    		\end{tabular}
    	\end{center}
   \end{table}
\end{center}
\subsection{JWebUnit test cases}
Is the same of \textit{Vulnerability 146} at page ~\pageref{V146}, but the exploit of the vulnerability is visibile through a \textit{teacher} account instead of a \textit{parent} one as in the former vulnerability.
\end{homeworkProblem}
\clearpage

%----------------------------------------------------------------------------------------
%	VULNERABILITY 149
%----------------------------------------------------------------------------------------

\begin{homeworkProblem}{Vulnerability 149}
\subsection{Brief Analysis}
\begin{center}
	File: EditUser.php
	\begin{table}[h]
		\begin{center}
    		\begin{tabular}{ | c | c | }
    			\hline
    			\textbf{VARIABLE} & \textbf{RESULT} \\ \hline
  				page & true \\ \hline
  				page2 & true \\ \hline
  				delete & true \\ \hline
    		\end{tabular}
    	\end{center}
   \end{table}
\end{center}
\subsection{JWebUnit test cases}
The code is adapted from the one of \textit{Vulnerability 37} at page ~\pageref{sec:V37}.
\end{homeworkProblem}
\clearpage

%----------------------------------------------------------------------------------------
%	VULNERABILITY 161
%----------------------------------------------------------------------------------------

\begin{homeworkProblem}{Vulnerability 161}
\subsection{Brief Analysis}
\begin{center}
	File: EditParent.php
	\begin{table}[h]
		\begin{center}
    		\begin{tabular}{ | c | c | }
    			\hline
    			\textbf{VARIABLE} & \textbf{RESULT} \\ \hline
  				page & true \\ \hline
  				page2 & true \\ \hline
  				delete & true \\ \hline
    		\end{tabular}
    	\end{center}
   \end{table}
\end{center}
\subsection{JWebUnit test cases}
The code is adapted from the one of \textit{Vulnerability 37} at page ~\pageref{sec:V37}.
\end{homeworkProblem}
\clearpage

%----------------------------------------------------------------------------------------
%	VULNERABILITY 165
%----------------------------------------------------------------------------------------

\begin{homeworkProblem}{Vulnerability 165}
\label{V165}
\subsection{Brief Analysis}
\begin{center}
	File: StudentMain.php
	\begin{table}[h]
		\begin{center}
    		\begin{tabular}{ | c | c | }
    			\hline
    			\textbf{VARIABLE} & \textbf{RESULT} \\ \hline
  				page & true \\ \hline
  				page2 & true \\ \hline
  				selectclass & true \\ \hline
    		\end{tabular}
    	\end{center}
   \end{table}
\end{center}
\subsection{JWebUnit test cases}
The code is adapted from the one of \textit{Vulnerability 11} at page ~\pageref{sec:V11}.
\end{homeworkProblem}
\clearpage

%----------------------------------------------------------------------------------------
%	VULNERABILITY 180
%----------------------------------------------------------------------------------------

\begin{homeworkProblem}{Vulnerability 180}
\subsection{Brief Analysis}
\begin{center}
	File: TeacherMain.php
	\begin{table}[h]
		\begin{center}
    		\begin{tabular}{ | c | c | }
    			\hline
    			\textbf{VARIABLE} & \textbf{RESULT} \\ \hline
  				page & true \\ \hline
  				page2 & true \\ \hline
  				selectclass & true \\ \hline
    		\end{tabular}
    	\end{center}
   \end{table}
\end{center}
\subsection{JWebUnit test cases}
Is the same of \textit{Vulnerability 165} at page ~\pageref{V165}, but the exploit of the vulnerability is visibile through a \textit{teacher} account instead of a \textit{student} one as in the former vulnerability.
\end{homeworkProblem}
\clearpage

%----------------------------------------------------------------------------------------
%	VULNERABILITY 181
%----------------------------------------------------------------------------------------

\begin{homeworkProblem}{Vulnerability 181}
\subsection{Brief Analysis}
\begin{center}
	File: ViewStudents.php
	\begin{table}[h]
		\begin{center}
    		\begin{tabular}{ | c | c | }
    			\hline
    			\textbf{VARIABLE} & \textbf{RESULT} \\ \hline
  				page & true \\ \hline
  				page2 & true \\ \hline
  				selectclass & true \\ \hline
    		\end{tabular}
    	\end{center}
   \end{table}
\end{center}
\subsection{JWebUnit test cases}
The code is adapted from the one of \textit{Vulnerability 11} at page ~\pageref{sec:V11}.
\end{homeworkProblem}
\clearpage

%----------------------------------------------------------------------------------------
%	VULNERABILITY 183
%----------------------------------------------------------------------------------------

\begin{homeworkProblem}{Vulnerability 183}
\label{V183}
\subsection{Brief Analysis}
\begin{center}
	File: ViewAssignments.php
	\begin{table}[h]
		\begin{center}
    		\begin{tabular}{ | c | c | }
    			\hline
    			\textbf{VARIABLE} & \textbf{RESULT} \\ \hline
  				page & true \\ \hline
  				page2 & true \\ \hline
  				selectclass & true \\ \hline
  				onpage & true \\ \hline
    		\end{tabular}
    	\end{center}
   \end{table}
\end{center}
\subsection{JWebUnit test cases}
The code is adapted from the one of \textit{Vulnerability 11} at page ~\pageref{sec:V11}.
\subsubsection{onpage}
\textbf{VERIFICARE}
\end{homeworkProblem}
\clearpage

%----------------------------------------------------------------------------------------
%	VULNERABILITY 184
%----------------------------------------------------------------------------------------

\begin{homeworkProblem}{Vulnerability 184}
\subsection{Brief Analysis}
\begin{center}
	File: ViewAssignments.php
	\begin{table}[h]
		\begin{center}
    		\begin{tabular}{ | c | c | }
    			\hline
    			\textbf{VARIABLE} & \textbf{RESULT} \\ \hline
  				page & true \\ \hline
  				page2 & true \\ \hline
  				selectclass & true \\ \hline
  				onpage & true \\ \hline
    		\end{tabular}
    	\end{center}
   \end{table}
\end{center}
\subsection{JWebUnit test cases}
Is the same of \textit{Vulnerability 183} at page ~\pageref{V183}, but the exploit of the vulnerability is visibile through a \textit{student} account instead of a \textit{parent} one as in the former vulnerability.
\textbf{VERIFICARE ONPAGE}
\end{homeworkProblem}
\clearpage

%----------------------------------------------------------------------------------------
%	VULNERABILITY 186
%----------------------------------------------------------------------------------------

\begin{homeworkProblem}{Vulnerability 186}
\subsection{Brief Analysis}
\begin{center}
	File: AdminMain.php
	\begin{table}[h]
		\begin{center}
    		\begin{tabular}{ | c | c | }
    			\hline
    			\textbf{VARIABLE} & \textbf{RESULT} \\ \hline
  				page & true \\ \hline
  				page2 & true \\ \hline
    		\end{tabular}
    	\end{center}
   \end{table}
\end{center}
\subsection{JWebUnit test cases}
The code is adapted from the one of \textit{Vulnerability 11} at page ~\pageref{sec:V11}.
\end{homeworkProblem}
\clearpage

%----------------------------------------------------------------------------------------
%	VULNERABILITY 191
%----------------------------------------------------------------------------------------

\begin{homeworkProblem}{Vulnerability 191}
\subsection{Brief Analysis}
\begin{center}
	File: DeficiencyReport.php
	\begin{table}[h]
		\begin{center}
    		\begin{tabular}{ | c | c | }
    			\hline
    			\textbf{VARIABLE} & \textbf{RESULT} \\ \hline
  				page & true \\ \hline
  				page2 & true \\ \hline
    		\end{tabular}
    	\end{center}
   \end{table}
\end{center}
\subsection{JWebUnit test cases}
The JWebUnit test cases of this vulnerability, were a bit different from the others, the access to the page is done through a \textit{select} with an \textit{onChange trigger}.
\subsubsection{page}
\begin{lstlisting}[language=Java,caption=jwebunit test code for \textit{page}]
	public void page(){
		Vulnerabilities.page(tester,"students",null);
		tester.selectOption("report","Deficiency Report");
		tester.assertMatch("Deficiency Report");
		tester.assertLinkNotPresentWithText("malicious");
	}
\end{lstlisting}
\subsubsection{page2}
\begin{lstlisting}[language=Java,caption=jwebunit test code for \textit{page},label=lst:V191P2]
	public void page2(){
		IElement mySelect = tester.getElementByXPath("//option[text()='Deficiency Report']");
		String optionValue = mySelect.getAttribute("value");
		mySelect.setAttribute("value",optionValue + "'><a href='http://www.unitn.it'>malicious</a><br'");
		tester.selectOption("report","Deficiency Report");
		tester.assertMatch("Deficiency Report");
		tester.assertLinkNotPresentWithText("malicious");
	}
\end{lstlisting}
The page2 test case took advantage of this part of the onChange attribute of the select item:
\begin{lstlisting}[language=PHP,caption=portion of the source code of the displayed page (ViewStudents)]
 <select name='report' onChange='document.students.page2.value=document.students.report.value;document.students.deletestudent.value=0;document.students.submit();'>
\end{lstlisting}
In particular, \textit{document.students.page2.value=document.students.report.value;}, give the possibility to inject the attack in the value of the select option, as can be seen in the Listing~\ref{lst:V191P2} from line 2 to 4.
\end{homeworkProblem}
\clearpage

%----------------------------------------------------------------------------------------
%	VULNERABILITY 194
%----------------------------------------------------------------------------------------

\begin{homeworkProblem}{Vulnerability 194}
\subsection{Brief Analysis}
\begin{center}
	File: ParentMain.php
	\begin{table}[h]
		\begin{center}
    		\begin{tabular}{ | c | c | }
    			\hline
    			\textbf{VARIABLE} & \textbf{RESULT} \\ \hline
  				page & true \\ \hline
  				page2 & true \\ \hline
  				selectclass & true \\ \hline
  				student & true \\ \hline
    		\end{tabular}
    	\end{center}
   \end{table}
\end{center}
\subsection{JWebUnit test cases}
The code is adapted from the one of \textit{Vulnerability 11} at page ~\pageref{sec:V11} and \textit{Vulnerability 13} at page ~\pageref{sec:V13}
\end{homeworkProblem}
\clearpage

%----------------------------------------------------------------------------------------
%	VULNERABILITY 200
%----------------------------------------------------------------------------------------

\begin{homeworkProblem}{Vulnerability 200}
\subsection{Brief Analysis}
\begin{center}
	File: ViewGrades.php
	\begin{table}[h]
		\begin{center}
    		\begin{tabular}{ | c | c | }
    			\hline
    			\textbf{VARIABLE} & \textbf{RESULT} \\ \hline
  				page & true \\ \hline
  				page2 & true \\ \hline
  				selectclass & true \\ \hline
    		\end{tabular}
    	\end{center}
   \end{table}
\end{center}
\subsection{JWebUnit test cases}
The code is adapted from the one of \textit{Vulnerability 11} at page ~\pageref{sec:V11}.
\end{homeworkProblem}
\clearpage

\end{document}
