%%%%%%%%%%%%%%%%%%%%%%%%%%%%%%%%%%%%%%%%%
%
% Author: Fabrizio Zeni
%
%%%%%%%%%%%%%%%%%%%%%%%%%%%%%%%%%%%%%%%%%

%----------------------------------------------------------------------------------------
%	PACKAGES AND OTHER DOCUMENT CONFIGURATIONS
%----------------------------------------------------------------------------------------

\documentclass{article}

\usepackage{fancyhdr} % Required for custom headers
\usepackage{lastpage} % Required to determine the last page for the footer
\usepackage{extramarks} % Required for headers and footers
\usepackage[usenames,dvipsnames]{color} % Required for custom colors
\usepackage{graphicx} % Required to insert images
\usepackage{listings} % Required for insertion of code
\usepackage{courier} % Required for the courier font
\usepackage{lipsum} % Used for inserting dummy 'Lorem ipsum' text into the template
\usepackage{caption}


\definecolor{mygreen}{rgb}{0,0.6,0}
\definecolor{mygray}{rgb}{0.5,0.5,0.5}
\definecolor{mymauve}{rgb}{0.58,0,0.82}


% Margins
\topmargin=-0.45in
\evensidemargin=0in
\oddsidemargin=0in
\textwidth=6.5in
\textheight=9.0in
\headsep=0.25in

\linespread{1.1} % Line spacing

% Set up the header and footer
\pagestyle{fancy}
\lhead{\hmwkAuthorName} % Top left header
\chead{\hmwkClass\ : \hmwkTitle} % Top center head
%\rhead{\firstxmark} % Top right header
\lfoot{\lastxmark} % Bottom left footer
\cfoot{} % Bottom center footer
\rfoot{Page\ \thepage\ of\ \protect\pageref{LastPage}} % Bottom right footer
\renewcommand\headrulewidth{0.4pt} % Size of the header rule
\renewcommand\footrulewidth{0.4pt} % Size of the footer rule

\setlength\parindent{0pt} % Removes all indentation from paragraphs

%----------------------------------------------------------------------------------------
%	CODE INCLUSION CONFIGURATION
%----------------------------------------------------------------------------------------

\definecolor{MyDarkGreen}{rgb}{0.0,0.4,0.0} % This is the color used for comments
\lstloadlanguages{PHP,Java,Sql} % Load Perl syntax for listings, for a list of other languages supported see: ftp://ftp.tex.ac.uk/tex-archive/macros/latex/contrib/listings/listings.pdf

\lstset{ %
  backgroundcolor=\color{SpringGreen},   % choose the background color; you must add \usepackage{color} or \usepackage{xcolor}
  basicstyle=\footnotesize,        % the size of the fonts that are used for the code
  breakatwhitespace=false,         % sets if automatic breaks should only happen at whitespace
  breaklines=true,                 % sets automatic line breaking
  captionpos=b,                    % sets the caption-position to bottom
  commentstyle=\color{mygreen},    % comment style
  deletekeywords={...},            % if you want to delete keywords from the given language
  escapeinside={\%*}{*)},          % if you want to add LaTeX within your code
  extendedchars=true,              % lets you use non-ASCII characters; for 8-bits encodings only, does not work with UTF-8
  frame=single,                    % adds a frame around the code
  keywordstyle=\color{blue},       % keyword style
  language=PHP,                 % the language of the code
  morekeywords={*,...},            % if you want to add more keywords to the set
  numbers=left,                    % where to put the line-numbers; possible values are (none, left, right)
  numbersep=5pt,                   % how far the line-numbers are from the code
  stepnumber=3,    
  firstnumber=1,
  numberfirstline=false
  %numberstyle=\tiny\color{gray}, % the style that is used for the line-numbers
  rulecolor=\color{black},         % if not set, the frame-color may be changed on line-breaks within not-black text (e.g. comments (green here))
  showspaces=false,                % show spaces everywhere adding particular underscores; it overrides 'showstringspaces'
  showstringspaces=false,          % underline spaces within strings only
  showtabs=false,                  % show tabs within strings adding particular underscores
  %stepnumber=2,                    % the step between two line-numbers. If it's 1, each line will be numbered
  stringstyle=\color{mymauve},     % string literal style
  tabsize=2,                       % sets default tabsize to 2 spaces
  title=\lstname                   % show the filename of files included with \lstinputlisting; also try caption instead of title
}


% Creates a new command to include a perl script, the first parameter is the filename of the script (without .pl), the second parameter is the caption
\newcommand{\phpscript}[1]{
\begin{itemize}
%\item[]\lstinputlisting[caption=#2,label=#2,firstline=#3,lastline=#4]{#1.php}
\item[]\lstinputlisting{/var/www/sm/#1.php}
\end{itemize}
}

%----------------------------------------------------------------------------------------
%	DOCUMENT STRUCTURE COMMANDS
%	Skip this unless you know what you're doing
%----------------------------------------------------------------------------------------

% Header and footer for when a page split occurs within a problem environment
\newcommand{\enterProblemHeader}[1]{
\nobreak\extramarks{#1}{#1 continued on next page\ldots}\nobreak
\nobreak\extramarks{#1 (continued)}{#1 continued on next page\ldots}\nobreak
}

% Header and footer for when a page split occurs between problem environments
\newcommand{\exitProblemHeader}[1]{
\nobreak\extramarks{#1 (continued)}{#1 continued on next page\ldots}\nobreak
\nobreak\extramarks{#1}{}\nobreak
}

\setcounter{secnumdepth}{0} % Removes default section numbers
\newcounter{homeworkProblemCounter} % Creates a counter to keep track of the number of problems

\newcommand{\homeworkProblemName}{}
\newenvironment{homeworkProblem}[1]{ % Makes a new environment called homeworkProblem which takes 1 argument (custom name) but the default is "Problem #"
%[Problem \arabic{homeworkProblemCounter}]
%\stepcounter{homeworkProblemCounter} % Increase counter for number of problems
\renewcommand{\homeworkProblemName}{#1} % Assign \homeworkProblemName the name of the problem
\begin{center}
\section{#1} % Make a section in the document with the custom problem count
\end{center}
\enterProblemHeader{#1} % Header and footer within the environment
%\exitProblemHeader{\homeworkProblemName}
}{
\exitProblemHeader{\homeworkProblemName} % Header and footer after the environment
}

\newcommand{\problemAnswer}[1]{ % Defines the problem answer command with the content as the only argument
\noindent\framebox[\columnwidth][c]{\begin{minipage}{0.98\columnwidth}#1\end{minipage}} % Makes the box around the problem answer and puts the content inside
}

\newcommand{\homeworkSectionName}{}
\newenvironment{homeworkSection}[1]{ % New environment for sections within homework problems, takes 1 argument - the name of the section
\renewcommand{\homeworkSectionName}{#1} % Assign \homeworkSectionName to the name of the section from the environment argument
\subsection{\homeworkSectionName} % Make a subsection with the custom name of the subsection
\enterProblemHeader{"asdasd"\ [\homeworkSectionName]}
%\enterProblemHeader{\homeworkProblemName\ [\homeworkSectionName]} % Header and footer within the environment
}
%{
%\enterProblemHeader{\homeworkProblemName} % Header and footer after the environment
%}

%----------------------------------------------------------------------------------------
%	NAME AND CLASS SECTION
%----------------------------------------------------------------------------------------

\newcommand{\hmwkTitle}{Assignment\ \#8} % Assignment title
\newcommand{\hmwkDueDate}{Friday,\ April\ 19,\ 2013} % Due date
\newcommand{\hmwkClass}{Security\ Testing} % Course/class
\newcommand{\hmwkClassTime}{13:30am} % Class/lecture time
\newcommand{\hmwkClassInstructor}{Jones} % Teacher/lecturer
\newcommand{\hmwkAuthorName}{Fabrizio\ Zeni} % Your name

%----------------------------------------------------------------------------------------
%	TITLE PAGE
%----------------------------------------------------------------------------------------

\title{
\vspace{2in}
\textmd{\textbf{\hmwkClass:\ \hmwkTitle}}\\
\normalsize\vspace{0.1in}\small{Due\ on\ \hmwkDueDate}\\
\vspace{0.1in}\large{\textit{\hmwkClassInstructor\ \hmwkClassTime}}
\vspace{3in}
}

\author{\textbf{\hmwkAuthorName}}
\date{} % Insert date here if you want it to appear below your name

%----------------------------------------------------------------------------------------

\begin{document}

\maketitle

%----------------------------------------------------------------------------------------
%	TABLE OF CONTENTS
%----------------------------------------------------------------------------------------

%\setcounter{tocdepth}{1} % Uncomment this line if you don't want subsections listed in the ToC

\newpage
\tableofcontents
\newpage

%----------------------------------------------------------------------------------------
%	VULNERABILITY 11
%----------------------------------------------------------------------------------------

\begin{homeworkProblem}{Vulnerability 11}
\subsection{Brief Analysis}
\begin{center}
	\includegraphics[scale=0.20]{/home/ashen/Scaricati/pixy/graphs/indexResults/11.jpg}
\end{center}
\begin{center}
	File: AddAssignment.php
	\begin{table}[h]
		\begin{center}
    		\begin{tabular}{ | c | c | c |}
    			\hline
    			\textbf{VARIABLE} & \textbf{RESULT} \\ \hline
  				page & true \\ \hline
  				page2 & true \\ \hline
  				selectclass & true \\ \hline
    		\end{tabular}
    	\end{center}
   \end{table}
\end{center}

\subsection{JWebUnit test cases}
\subsubsection{prepare and cleanup}
\lstinputlisting[firstline=16,lastline=25,language=Java,caption=prepare function]{/home/ashen/workspace/TestingFramework/src/tests/V11.java}
\lstinputlisting[firstline=49,lastline=52,language=Java,caption=cleanup function]{/home/ashen/workspace/TestingFramework/src/tests/V11.java}
In these two functions there is nothing special, just navigation and call to the login/logout utilities.
\subsubsection{page}
\lstinputlisting[firstline=28,lastline=32,language=Java,caption=jwebunit test code for \textit{page}]{/home/ashen/workspace/TestingFramework/src/tests/V11.java}
\lstinputlisting[firstline=7,lastline=13,language=Java,caption=function for the \textit{page} vulnerability]{/home/ashen/workspace/TestingFramework/src/util/Vulnerabilities.java}
This code does the test for \textit{page}. In order to catch the correct hidden field it was necessary to filter the form first, because there were two hidden fields with the same name and the first is not the one triggered by the buttons.
So the function retrieves the page2 input element and stores it into the \emph{oldValue} variable, which at line 6 is concatenated to the malicious link and inserted into the page value.
\subsubsection{page2}
\lstinputlisting[firstline=35,lastline=39,language=Java,caption=jwebunit test code for \textit{page2}]{/home/ashen/workspace/TestingFramework/src/tests/V11.java}
\lstinputlisting[firstline=15,lastline=23,language=Java,caption=function for the \textit{page2} vulnerability]{/home/ashen/workspace/TestingFramework/src/util/Vulnerabilities.java}
The page2 vulnerability was more subtle to automatically trigger. That was due to the fact that the form buttons have a \textit{javascript} code in the attribute \textbf{onClick}, which write on the page2 value. So that in order to prevent the button from modify the injected value, at line 3 the button element is retrieved, then we get the value of the onClick attribute, which is processed by the \emph{page2Fix function} -  which purge the attribute from any command that modifies the page2 value and returns the value for page2 and the other instructions that need to be put back into the attribute.
\subsubsection{selectclass}
\lstinputlisting[firstline=42,lastline=46,language=Java,caption=jwebunit test code for \textit{selectclass}]{/home/ashen/workspace/TestingFramework/src/tests/V11.java}
\lstinputlisting[firstline=34,lastline=39,language=Java,caption=function for the \textit{selectclass} vulnerability]{/home/ashen/workspace/TestingFramework/src/util/Vulnerabilities.java}
The selectclass vulnerability was almost straightforward and differs from the \textit{page} function just in the attribute name in the XPath expression. 
\end{homeworkProblem}
\clearpage

%----------------------------------------------------------------------------------------
%	VULNERABILITY 13
%----------------------------------------------------------------------------------------

\begin{homeworkProblem}{Vulnerability 13}
\subsection{Brief Analysis}
\begin{center}
	\includegraphics[scale=0.29]{/home/ashen/Scaricati/pixy/graphs/indexResults/13.jpg}
\end{center}
\begin{center}
	File: AddAttendance.php
	\begin{table}[h]
		\begin{center}
    		\begin{tabular}{ | c | c | c |}
    			\hline
    			\textbf{VARIABLE} & \textbf{RESULT} \\ \hline
  				page & true \\ \hline
  				page2 & true \\ \hline
  				student & true \\ \hline
  				semester & true \\ \hline
    		\end{tabular}
    	\end{center}
   \end{table}
\end{center}
\clearpage
\subsection{JWebUnit test cases}
\subsubsection{prepare and cleanup}
\lstinputlisting[firstline=16,lastline=23,language=Java,caption=prepare function]{/home/ashen/workspace/TestingFramework/src/tests/V13.java}
\lstinputlisting[firstline=54,lastline=57,language=Java,caption=cleanup function]{/home/ashen/workspace/TestingFramework/src/tests/V13.java}
\subsubsection{page}
\lstinputlisting[firstline=26,lastline=30,language=Java,caption=jwebunit test code for \textit{page}]{/home/ashen/workspace/TestingFramework/src/tests/V13.java}
\subsubsection{page2}
\lstinputlisting[firstline=33,lastline=37,language=Java,caption=jwebunit test code for \textit{page2}]{/home/ashen/workspace/TestingFramework/src/tests/V13.java}
\clearpage
\subsubsection{student}
\lstinputlisting[firstline=40,lastline=44,language=Java,caption=jwebunit test code for \textit{student}]{/home/ashen/workspace/TestingFramework/src/tests/V13.java}
\lstinputlisting[firstline=51,lastline=57,language=Java,caption=function for vulnerabilities over select input elements]{/home/ashen/workspace/TestingFramework/src/util/Vulnerabilities.java}
In this case the input element was a \textbf{select}, so the XPATH expression was modified with \emph{//option[@selected]} to catch the selected option. The remaining part of the code is almost equivalent to the \textit{page} one.
\subsubsection{semester}
\lstinputlisting[firstline=47,lastline=51,language=Java,caption=jwebunit test code for \textit{semester}]{/home/ashen/workspace/TestingFramework/src/tests/V13.java}
The semester test is a copy-paste of the student one.
\end{homeworkProblem}
\clearpage

%----------------------------------------------------------------------------------------
%	VULNERABILITY 16
%----------------------------------------------------------------------------------------

\begin{homeworkProblem}{Vulnerability 16}
\subsection{Brief Analysis}
\begin{center}
	\includegraphics[scale=0.30]{/home/ashen/Scaricati/pixy/graphs/indexResults/16.jpg}
\end{center}
\begin{center}
	File: AddAnnouncements.php
	\begin{table}[h]
		\begin{center}
    		\begin{tabular}{ | c | c | c |}
    			\hline
    			\textbf{VARIABLE} & \textbf{RESULT} \\ \hline
  				page & true \\ \hline
  				page2 & true \\ \hline
    		\end{tabular}
    	\end{center}
   \end{table}
\end{center}
\clearpage
\subsection{JWebUnit test cases}
\subsubsection{prepare and cleanup}
\lstinputlisting[firstline=16,lastline=23,language=Java,caption=prepare function]{/home/ashen/workspace/TestingFramework/src/tests/V16.java}
\lstinputlisting[firstline=40,lastline=43,language=Java,caption=cleanup function]{/home/ashen/workspace/TestingFramework/src/tests/V16.java}
\subsubsection{page}
\lstinputlisting[firstline=26,lastline=30,language=Java,caption=jwebunit test code for \textit{page}]{/home/ashen/workspace/TestingFramework/src/tests/V16.java}
\subsubsection{page2}
\lstinputlisting[firstline=33,lastline=37,language=Java,caption=jwebunit test code for \textit{page2}]{/home/ashen/workspace/TestingFramework/src/tests/V16.java}
\end{homeworkProblem}
\clearpage

%----------------------------------------------------------------------------------------
%	VULNERABILITY 18
%----------------------------------------------------------------------------------------

\begin{homeworkProblem}{Vulnerability 18}
\subsection{Brief Analysis}
\begin{center}
	\includegraphics[scale=0.30]{/home/ashen/Scaricati/pixy/graphs/indexResults/18.jpg}
\end{center}
\begin{center}
	File: AddUser.php
	\begin{table}[h]
		\begin{center}
    		\begin{tabular}{ | c | c | c |}
    			\hline
    			\textbf{VARIABLE} & \textbf{RESULT} \\ \hline
  				page & true \\ \hline
  				page2 & true \\ \hline
    		\end{tabular}
    	\end{center}
   \end{table}
\end{center}
\clearpage
\subsection{JWebUnit test cases}
\subsubsection{prepare and cleanup}
\lstinputlisting[firstline=16,lastline=23,language=Java,caption=prepare function]{/home/ashen/workspace/TestingFramework/src/tests/V18.java}
\lstinputlisting[firstline=40,lastline=43,language=Java,caption=cleanup function]{/home/ashen/workspace/TestingFramework/src/tests/V18.java}
\subsubsection{page}
\lstinputlisting[firstline=26,lastline=30,language=Java,caption=jwebunit test code for \textit{page}]{/home/ashen/workspace/TestingFramework/src/tests/V18.java}
\subsubsection{page2}
\lstinputlisting[firstline=33,lastline=37,language=Java,caption=jwebunit test code for \textit{page2}]{/home/ashen/workspace/TestingFramework/src/tests/V18.java}
\end{homeworkProblem}
\clearpage

%----------------------------------------------------------------------------------------
%	VULNERABILITY 19
%----------------------------------------------------------------------------------------

\begin{homeworkProblem}{Vulnerability 19}
\subsection{Brief Analysis}
\begin{center}
	\includegraphics[scale=0.35]{/home/ashen/Scaricati/pixy/graphs/indexResults/19.jpg}
\end{center}
\begin{center}
	File: AddTerm.php
	\begin{table}[h]
		\begin{center}
    		\begin{tabular}{ | c | c | c |}
    			\hline
    			\textbf{VARIABLE} & \textbf{RESULT} \\ \hline
  				page & true \\ \hline
  				page2 & true \\ \hline
    		\end{tabular}
    	\end{center}
   \end{table}
\end{center}
\clearpage
\subsection{JWebUnit test cases}
\subsubsection{prepare and cleanup}
\lstinputlisting[firstline=17,lastline=24,language=Java,caption=prepare function]{/home/ashen/workspace/TestingFramework/src/tests/V19.java}
\lstinputlisting[firstline=41,lastline=44,language=Java,caption=cleanup function]{/home/ashen/workspace/TestingFramework/src/tests/V19.java}
\subsubsection{page}
\lstinputlisting[firstline=27,lastline=31,language=Java,caption=jwebunit test code for \textit{page}]{/home/ashen/workspace/TestingFramework/src/tests/V19.java}
\subsubsection{page2}
\lstinputlisting[firstline=34,lastline=38,language=Java,caption=jwebunit test code for \textit{page2}]{/home/ashen/workspace/TestingFramework/src/tests/V19.java}
\end{homeworkProblem}
\clearpage

%----------------------------------------------------------------------------------------
%	VULNERABILITY 30,31
%----------------------------------------------------------------------------------------

\begin{homeworkProblem}{Vulnerability 30,31}
\subsection{Brief Analysis}
\begin{center}
	\includegraphics[scale=0.35]{/home/ashen/Scaricati/pixy/graphs/indexResults/30.jpg}
\end{center}
\begin{center}
	File: ViewAssignments.php
	\begin{table}[h]
		\begin{center}
    		\begin{tabular}{ | c | c | c |}
    			\hline
    			\textbf{VARIABLE} & \textbf{RESULT} \\ \hline
  				page & true \\ \hline
  				page2 & true \\ \hline
    		\end{tabular}
    	\end{center}
   \end{table}
\end{center}
\clearpage
\subsection{JWebUnit test cases}
\subsubsection{prepare and cleanup}
\lstinputlisting[firstline=16,lastline=23,language=Java,caption=prepare function]{/home/ashen/workspace/TestingFramework/src/tests/V30.java}
\lstinputlisting[firstline=43,lastline=46,language=Java,caption=cleanup function]{/home/ashen/workspace/TestingFramework/src/tests/V30.java}
\subsubsection{page}
\lstinputlisting[firstline=26,lastline=32,language=Java,caption=jwebunit test code for \textit{page}]{/home/ashen/workspace/TestingFramework/src/tests/V30.java}
\subsubsection{page2}
\lstinputlisting[firstline=35,lastline=40,language=Java,caption=jwebunit test code for \textit{page2}]{/home/ashen/workspace/TestingFramework/src/tests/V30.java}
\lstinputlisting[firstline=25,lastline=32,language=Java,caption=function for the page2 vulnerability with links]{/home/ashen/workspace/TestingFramework/src/util/Vulnerabilities.java}
Here a modified version of the page2 utility function is used. That is due to the fact that in this case we have to modify a link instead of a button.
\end{homeworkProblem}
\clearpage

%----------------------------------------------------------------------------------------
%	VULNERABILITY 37
%----------------------------------------------------------------------------------------

\begin{homeworkProblem}{Vulnerability 37}
\subsection{Brief Analysis}
\begin{center}
	\includegraphics[scale=0.18]{/home/ashen/Scaricati/pixy/graphs/indexResults/37.jpg}
\end{center}
\begin{center}
	File: EditAssignment.php
	\begin{table}[h]
		\begin{center}
    		\begin{tabular}{ | c | c | c |}
    			\hline
    			\textbf{VARIABLE} & \textbf{RESULT} \\ \hline
  				page & true \\ \hline
  				page2 & true \\ \hline
  				selectclass & true \\ \hline
  				delete & true \\ \hline
    		\end{tabular}
    	\end{center}
   \end{table}
\end{center}
\clearpage
\subsection{JWebUnit test cases}
\subsubsection{prepare and cleanup}
\lstinputlisting[firstline=17,lastline=30,language=Java,caption=prepare function]{/home/ashen/workspace/TestingFramework/src/tests/V37.java}
The prepare functions was a bit longer this time, because in order to access to the reported page one of the assignment has to be checked in the checkbox element. This is done by retrieving the line of the assignment \emph{prova} and finally we set insert in the \textit{delete[]} the value of the selected assignment.
\lstinputlisting[firstline=61,lastline=64,language=Java,caption=cleanup function]{/home/ashen/workspace/TestingFramework/src/tests/V37.java}
\subsubsection{page}
\lstinputlisting[firstline=33,lastline=37,language=Java,caption=jwebunit test code for \textit{page}]{/home/ashen/workspace/TestingFramework/src/tests/V37.java}
\subsubsection{page2}
\lstinputlisting[firstline=40,lastline=44,language=Java,caption=jwebunit test code for \textit{page2}]{/home/ashen/workspace/TestingFramework/src/tests/V37.java}
\subsubsection{selectclass}
\lstinputlisting[firstline=47,lastline=51,language=Java,caption=jwebunit test code for \textit{selectclass}]{/home/ashen/workspace/TestingFramework/src/tests/V37.java}
\subsubsection{delete}
\lstinputlisting[firstline=54,lastline=58,language=Java,caption=jwebunit test code for \textit{delete}]{/home/ashen/workspace/TestingFramework/src/tests/V37.java}
\lstinputlisting[firstline=41,lastline=49,language=Java,caption=function for the \textit{delete} vulnerability]{/home/ashen/workspace/TestingFramework/src/util/Vulnerabilities.java}
The interesting thing of this case is that even a \emph{sql injection} is possible by putting another query after the semicolon.
\end{homeworkProblem}
\clearpage

%----------------------------------------------------------------------------------------
%	VULNERABILITY 41
%----------------------------------------------------------------------------------------

\begin{homeworkProblem}{Vulnerability 41}
\subsection{Brief Analysis}
\begin{center}
	\includegraphics[scale=0.29]{/home/ashen/Scaricati/pixy/graphs/indexResults/41.jpg}
\end{center}
\begin{center}
	File: EditAnnouncement.php
	\begin{table}[h]
		\begin{center}
    		\begin{tabular}{ | c | c | c |}
    			\hline
    			\textbf{VARIABLE} & \textbf{RESULT} \\ \hline
  				page & true \\ \hline
  				page2 & true \\ \hline
  				delete & true \\ \hline
    		\end{tabular}
    	\end{center}
   \end{table}
\end{center}
\clearpage
\subsection{JWebUnit test cases}
\subsubsection{prepare and cleanup}
\lstinputlisting[firstline=17,lastline=27,language=Java,caption=prepare function]{/home/ashen/workspace/TestingFramework/src/tests/V41.java}
\lstinputlisting[firstline=51,lastline=54,language=Java,caption=cleanup function]{/home/ashen/workspace/TestingFramework/src/tests/V41.java}
\subsubsection{page}
\lstinputlisting[firstline=30,lastline=34,language=Java,caption=jwebunit test code for \textit{page}]{/home/ashen/workspace/TestingFramework/src/tests/V41.java}
\subsubsection{page2}
\lstinputlisting[firstline=37,lastline=41,language=Java,caption=jwebunit test code for \textit{page2}]{/home/ashen/workspace/TestingFramework/src/tests/V41.java}
\subsubsection{delete}
\lstinputlisting[firstline=44,lastline=48,language=Java,caption=jwebunit test code for \textit{delete}]{/home/ashen/workspace/TestingFramework/src/tests/V41.java}
\end{homeworkProblem}
\clearpage

%----------------------------------------------------------------------------------------
%	VULNERABILITY 44
%----------------------------------------------------------------------------------------

\begin{homeworkProblem}{Vulnerability 44}
\subsection{Brief Analysis}
\begin{center}
	\includegraphics[scale=0.29]{/home/ashen/Scaricati/pixy/graphs/indexResults/44.jpg}
\end{center}
\begin{center}
	File: EditTerm.php
	\begin{table}[h]
		\begin{center}
    		\begin{tabular}{ | c | c | c |}
    			\hline
    			\textbf{VARIABLE} & \textbf{RESULT} \\ \hline
  				page & true \\ \hline
  				page2 & true \\ \hline
  				delete & true \\ \hline
    		\end{tabular}
    	\end{center}
   \end{table}
\end{center}
\clearpage
\subsection{JWebUnit test cases}
\subsubsection{prepare and cleanup}
\lstinputlisting[firstline=17,lastline=27,language=Java,caption=prepare function]{/home/ashen/workspace/TestingFramework/src/tests/V44.java}
\lstinputlisting[firstline=51,lastline=54,language=Java,caption=cleanup function]{/home/ashen/workspace/TestingFramework/src/tests/V44.java}
\subsubsection{page}
\lstinputlisting[firstline=30,lastline=34,language=Java,caption=jwebunit test code for \textit{page}]{/home/ashen/workspace/TestingFramework/src/tests/V44.java}
\subsubsection{page2}
\lstinputlisting[firstline=37,lastline=41,language=Java,caption=jwebunit test code for \textit{page2}]{/home/ashen/workspace/TestingFramework/src/tests/V44.java}
\subsubsection{delete}
\lstinputlisting[firstline=44,lastline=48,language=Java,caption=jwebunit test code for \textit{delete}]{/home/ashen/workspace/TestingFramework/src/tests/V44.java}
\end{homeworkProblem}
\clearpage

%----------------------------------------------------------------------------------------
%	VULNERABILITY 54
%----------------------------------------------------------------------------------------

\begin{homeworkProblem}{Vulnerability 54}
\subsection{Brief Analysis}
\begin{center}
	\includegraphics[scale=0.37]{/home/ashen/Scaricati/pixy/graphs/indexResults/54.jpg}
\end{center}
\begin{center}
	File: Login.php
	\begin{table}[h]
		\begin{center}
    		\begin{tabular}{ | c | c | c |}
    			\hline
    			\textbf{VARIABLE} & \textbf{RESULT} \\ \hline
  				text & true \\ \hline
    		\end{tabular}
    	\end{center}
   \end{table}
\end{center}
\clearpage
\subsection{JWebUnit test cases}
\subsubsection{prepare and cleanup}
\lstinputlisting[firstline=12,lastline=19,language=Java,caption=prepare function]{/home/ashen/workspace/TestingFramework/src/tests/V54.java}
\lstinputlisting[firstline=31,lastline=40,language=Java,caption=cleanup function]{/home/ashen/workspace/TestingFramework/src/tests/V54.java}
\subsubsection{page}
\lstinputlisting[firstline=22,lastline=28,language=Java,caption=jwebunit test code for \textit{text}]{/home/ashen/workspace/TestingFramework/src/tests/V54.java}
\end{homeworkProblem}
\clearpage

%----------------------------------------------------------------------------------------
%	VULNERABILITY 63
%----------------------------------------------------------------------------------------

\begin{homeworkProblem}{Vulnerability 63}
\subsection{Brief Analysis}
\begin{center}
	\includegraphics[scale=0.35]{/home/ashen/Scaricati/pixy/graphs/indexResults/63.jpg}
\end{center}
\begin{center}
	File: ViewAssignments.php
	\begin{table}[h]
		\begin{center}
    		\begin{tabular}{ | c | c | c |}
    			\hline
    			\textbf{VARIABLE} & \textbf{RESULT} \\ \hline
  				page & true \\ \hline
  				page2 & true \\ \hline
    		\end{tabular}
    	\end{center}
   \end{table}
\end{center}
\clearpage
\subsection{JWebUnit test cases}
\subsubsection{prepare and cleanup}
\lstinputlisting[firstline=16,lastline=23,language=Java,caption=prepare function]{/home/ashen/workspace/TestingFramework/src/tests/V63.java}
\lstinputlisting[firstline=40,lastline=43,language=Java,caption=cleanup function]{/home/ashen/workspace/TestingFramework/src/tests/V63.java}
\subsubsection{page}
\lstinputlisting[firstline=26,lastline=30,language=Java,caption=jwebunit test code for \textit{page}]{/home/ashen/workspace/TestingFramework/src/tests/V63.java}
\subsubsection{page2}
\lstinputlisting[firstline=33,lastline=37,language=Java,caption=jwebunit test code for \textit{page2}]{/home/ashen/workspace/TestingFramework/src/tests/V63.java}
\end{homeworkProblem}
\clearpage

%----------------------------------------------------------------------------------------
%	VULNERABILITY 70
%----------------------------------------------------------------------------------------

\begin{homeworkProblem}{Vulnerability 70}
\subsection{Brief Analysis}
\begin{center}
	\includegraphics[scale=0.35]{/home/ashen/Scaricati/pixy/graphs/indexResults/70.jpg}
\end{center}
\begin{center}
	File: ViewAssignments.php
	\begin{table}[h]
		\begin{center}
    		\begin{tabular}{ | c | c | c |}
    			\hline
    			\textbf{VARIABLE} & \textbf{RESULT} \\ \hline
  				page & true \\ \hline
  				page2 & true \\ \hline
    		\end{tabular}
    	\end{center}
   \end{table}
\end{center}
\clearpage
\subsection{JWebUnit test cases}
\subsubsection{prepare and cleanup}
\lstinputlisting[firstline=16,lastline=23,language=Java,caption=prepare function]{/home/ashen/workspace/TestingFramework/src/tests/V70.java}
\lstinputlisting[firstline=40,lastline=43,language=Java,caption=cleanup function]{/home/ashen/workspace/TestingFramework/src/tests/V70.java}
\subsubsection{page}
\lstinputlisting[firstline=26,lastline=30,language=Java,caption=jwebunit test code for \textit{page}]{/home/ashen/workspace/TestingFramework/src/tests/V70.java}
\subsubsection{page2}
\lstinputlisting[firstline=33,lastline=37,language=Java,caption=jwebunit test code for \textit{page2}]{/home/ashen/workspace/TestingFramework/src/tests/V70.java}
\end{homeworkProblem}
\clearpage

%----------------------------------------------------------------------------------------
%	VULNERABILITY 71
%----------------------------------------------------------------------------------------

\begin{homeworkProblem}{Vulnerability 71}
\subsection{Brief Analysis}
\begin{center}
	\includegraphics[scale=0.35]{/home/ashen/Scaricati/pixy/graphs/indexResults/71.jpg}
\end{center}
\begin{center}
	File: ViewAssignments.php
	\begin{table}[h]
		\begin{center}
    		\begin{tabular}{ | c | c | c |}
    			\hline
    			\textbf{VARIABLE} & \textbf{RESULT} \\ \hline
  				page & true \\ \hline
  				page2 & true \\ \hline
    		\end{tabular}
    	\end{center}
   \end{table}
\end{center}
\clearpage
\subsection{JWebUnit test cases}
\subsubsection{prepare and cleanup}
\lstinputlisting[firstline=16,lastline=23,language=Java,caption=prepare function]{/home/ashen/workspace/TestingFramework/src/tests/V71.java}
\lstinputlisting[firstline=40,lastline=43,language=Java,caption=cleanup function]{/home/ashen/workspace/TestingFramework/src/tests/V71.java}
\subsubsection{page}
\lstinputlisting[firstline=26,lastline=30,language=Java,caption=jwebunit test code for \textit{page}]{/home/ashen/workspace/TestingFramework/src/tests/V71.java}
\subsubsection{page2}
\lstinputlisting[firstline=33,lastline=37,language=Java,caption=jwebunit test code for \textit{page2}]{/home/ashen/workspace/TestingFramework/src/tests/V71.java}
\end{homeworkProblem}
\clearpage

%----------------------------------------------------------------------------------------
%	VULNERABILITY 76
%----------------------------------------------------------------------------------------

\begin{homeworkProblem}{Vulnerability 76}
\subsection{Brief Analysis}
\begin{center}
	\includegraphics[scale=0.18]{/home/ashen/Scaricati/pixy/graphs/indexResults/76.jpg}
\end{center}
\begin{center}
	File: EditAnnouncement.php
	\begin{table}[h]
		\begin{center}
    		\begin{tabular}{ | c | c | c |}
    			\hline
    			\textbf{VARIABLE} & \textbf{RESULT} \\ \hline
  				page & true \\ \hline
  				page2 & true \\ \hline
  				selectclass & true \\ \hline
  				assignment & true \\ \hline
  				delete & true \\ \hline
    		\end{tabular}
    	\end{center}
   \end{table}
\end{center}
\clearpage
\subsection{JWebUnit test cases}
\subsubsection{prepare and cleanup}
\lstinputlisting[firstline=17,lastline=29,language=Java,caption=prepare function]{/home/ashen/workspace/TestingFramework/src/tests/V76.java}
\lstinputlisting[firstline=68,lastline=71,language=Java,caption=cleanup function]{/home/ashen/workspace/TestingFramework/src/tests/V76.java}
\subsubsection{page}
\lstinputlisting[firstline=32,lastline=36,language=Java,caption=jwebunit test code for \textit{page}]{/home/ashen/workspace/TestingFramework/src/tests/V76.java}
\subsubsection{page2}
\lstinputlisting[firstline=39,lastline=43,language=Java,caption=jwebunit test code for \textit{page2}]{/home/ashen/workspace/TestingFramework/src/tests/V76.java}
\subsubsection{selectclass}
\lstinputlisting[firstline=46,lastline=50,language=Java,caption=jwebunit test code for \textit{selectclass}]{/home/ashen/workspace/TestingFramework/src/tests/V76.java}
\subsubsection{assignment}
\lstinputlisting[firstline=53,lastline=57,language=Java,caption=jwebunit test code for \textit{assignment}]{/home/ashen/workspace/TestingFramework/src/tests/V76.java}
\lstinputlisting[firstline=4,lastline=4,caption=EditGrade.php read of assignment]{/var/www/sm/EditGrade.php}
In this case, the input element is a \textit{select}, but the posted variable is printed inside an sql query - so as already said for \emph{Vulnerability 37} - an Sql Injection is also possible.
\subsubsection{delete}
\lstinputlisting[firstline=60,lastline=65,language=Java,caption=jwebunit test code for \textit{delete}]{/home/ashen/workspace/TestingFramework/src/tests/V76.java}
\end{homeworkProblem}
\clearpage

%----------------------------------------------------------------------------------------
%	VULNERABILITIES 11,13,16,18,19,37,41,44,63,70,71,76,85,87,88,89,90,93,111,115,126,
%					138,141,142,146,147,148,149,161,165,180,181,183,184,191,194,200,201
%					212,230,238,239,241,257,260,268,272,273,283,288,293,299,309,316,320
%----------------------------------------------------------------------------------------
\begin{homeworkProblem}{Vulnerabilities\textsuperscript{(*)}}
%\problemAnswer{
%\lipsum[3-5]
%}
\subsection{Brief Analysis}
\begin{center}
	\begin{table}[h]
	%caption{Example table}
		\begin{center}
    		\begin{tabular}{ | c | c | c |}
    			\hline
    			\textbf{VARIABLE} & \textbf{AFFECTED PAGES\textsuperscript{(*)}} & \textbf{RESULT} \\ \hline
  				page & all & true \\ \hline
  				page2 & all & positive \\ \hline
  				selectclass & 11,37,76,87,89,165,180,181,183,194,200,201,309,316 & positive \\ \hline
  				student & 13,142,194 & positive \\ \hline
  				semester & 13 & positive \\ \hline
  				delete & 37,41,44,76,85,111,115,149,161 & positive \\ \hline
  				assignment & 76 & positive \\ \hline
  				onpage & 146,183,257,260,268,273,283,288,293,309,320 & positive \\ \hline
   		 	\end{tabular}
    	\end{center}
    \caption*{\begin{tiny}\textsuperscript{(*)}11: AddAssignment.php | 13: AddAttendance.php |16: AddAnnouncements.php | 18: AddUser.php | 19: AddTerm.php | 37: EditAssignment.php | 41: EditAnnouncements.php | 44: EditTerms.php | 63: AddTeacher.php | 70: AddStudent.php | 71: AddSemester.php | 76: EditGrade.php | 85: EditSemester.php | 87/88: ViewClassSettings.php | 90: ViewStudents.php | 93: AddParent.php | 111: EditTeacher.php | 115: EditStudent.php | 126: ViewCourses.php | 138: StudentViewCourses.php | 141: AddClass.php | 142: ParentViewCourses.php | 146/147/148: ViewAnnoucements.php | 149: EditUser.php | 161: EditParent.php | 165: StudentMain.php | 180: TeacherMain.php | 181: ViewStudents.php | 183/184: ViewAssignments.php | 186/241: AdminMain.php | 191: DeficiencyReport.php | 194: ParentMain.php | 200/201: ViewGrades.php | 212: PointsReport.php | 130: VisualizeClasses.php | 238: VisualizeRegistration.php | 239: EditClasses.php | ManageAnnouncements.php | 260: ManageTerms.php | 268. ManageTerms.php | 272: ManageAttendance.php | 273: ManageTeachers.php | ManageUsers.php | 288: ManageParents.php | 293: ManageStudents.php | 299: Registration.php | 309: ManageAssignments.php | 316: ManageGrades.php | 320: ManageClasses.php\end{tiny}}
   \end{table}
\end{center}
\subsection{Explanation}
These parameters are used to process the web-application flow through. The problem is that the page which receive these values through a POST, do not validate them and they are put inside a the \textit{value} of a \textit{input} element.
\subsubsection{page}
\lstinputlisting[firstline=36,lastline=36,caption=AddAssignment.php load of \textit{page}]{/var/www/sm/AddAssignment.php}
\subsubsection{page2}
\lstinputlisting[firstline=33,lastline=33,caption=AddAssignment.php load of \textit{page2}]{/var/www/sm/AddAssignment.php}
\subsubsection{selectclass}
\lstinputlisting[firstline=35,lastline=35,caption=AddAssignment.php load of \textit{selectclass}]{/var/www/sm/AddAssignment.php}
\subsubsection{student}
\lstinputlisting[firstline=33,lastline=33,caption=AddAttendance.php load of \textit{student}]{/var/www/sm/AddAttendance.php}
\subsubsection{semester}
\lstinputlisting[firstline=32,lastline=32,caption=AddAttendance.php load of \textit{student}]{/var/www/sm/AddAttendance.php}
\subsubsection{delete}
\lstinputlisting[firstline=2,lastline=2,caption=EditAssignment.php load of \textit{delete}]{/var/www/sm/EditAssignment.php}
\lstinputlisting[firstline=43,lastline=43,caption=EditAssignment.php read of \textit{delete}]{/var/www/sm/EditAssignment.php}
\subsubsection{assignment}
\lstinputlisting[firstline=52,lastline=52,caption=EditGrade.php load of \textit{assignment}]{/var/www/sm/EditGrade.php}
\subsubsection{onpage}
\lstinputlisting[firstline=69,lastline=69,caption=ViewAnnouncements.php load of \textit{onpage}]{/var/www/sm/ViewAnnouncements.php}
\end{homeworkProblem}
\clearpage

%----------------------------------------------------------------------------------------
%	VULNERABILITY 30,31,207
%----------------------------------------------------------------------------------------
\begin{homeworkProblem}{Vulnerabilities 30,31,207}

\subsection{Brief Analysis}
\begin{center}
	\includegraphics[scale=0.20]{/home/ashen/Scaricati/pixy/graphs/indexResults/30.jpg}
\end{center}
\begin{center}
	Files: ViewAssignments.php,ManageAssignments.php
	\begin{table}[h]
	%caption{Example table}
		\begin{center}
    		\begin{tabular}{ | c | c | c |}
    			\hline
    			\textbf{VARIABLE} & \textbf{RESULT} \\ \hline
  				coursename & false positive \\ \hline
				assignment[5] & positive \\ \hline  				
    		\end{tabular}
    	\end{center}
   \end{table}
\end{center}
\subsection{Explanation}
\subsubsection{coursename}
In ManageClasses we have the 3 queries which do \textit{insertion} and one which do an \textit{update} inside the database table:

\lstinputlisting[firstline=23,lastline=24,caption=ManageClasses.php insert1 of \textit{coursename}]{/var/www/sm/ManageClasses.php}
\lstinputlisting[firstline=39,lastline=40,caption=ManageClasses.php insert2 of \textit{coursename}]{/var/www/sm/ManageClasses.php}
\lstinputlisting[firstline=46,lastline=47,caption=ManageClasses.php insert3 of \textit{coursename}]{/var/www/sm/ManageClasses.php}
\lstinputlisting[firstline=73,lastline=74,caption=ManageClasses.php update of \textit{coursename}]{/var/www/sm/ManageClasses.php}

No sanitization is made over the \$\_POST[title], so an xss can be injected.

In \textit{ViewAssignments.php} and \textit{ManageAssignments.php} we have the read of the tainted value: 
\lstinputlisting[firstline=6,lastline=7,caption=ViewAssignment.php load of \textit{coursename}]{/var/www/sm/ViewAssignments.php}
\lstinputlisting[firstline=6,lastline=7,caption=ManageAssignments.php load of \textit{coursename}]{/var/www/sm/ManageAssignments.php}

So far it seems legit to say that an XSS attack can be done over this vulnerability, having a look to the forms which are the source of the insertions and updates, we can see that a limit of 20 chars is set for the field:
\lstinputlisting[firstline=19,lastline=19,caption=AddClass.php form field]{/var/www/sm/AddClass.php}
\lstinputlisting[firstline=25,lastline=25,caption=EditClass.php form field]{/var/www/sm/EditClass.php}
Anyway we know that such restriction can be by-passed by intercepting the requests and modify them on-the-fly. However a closer look to the database structure denies our expectation and explains why this result is a false positive:
\lstinputlisting[firstline=55,lastline=55,caption=Sql structure of the field]{/var/www/sm/SchoolMate.sql}
In fact, the filed coursename\textit{•} is restricted to 20 chars even in the database structure and so any larger string is going to be truncated to that size. So no XSS can be injected because the shorter one that we know (\emph{\textit{<script>}alert('')</script>}) is \textbf{26} chars long.
\subsubsection{assignment[5]}
The source of this vulnerability is the value of the column \textit{assignmentinformation} of the table \textit{assignments}. The page \emph{ManageAssignments.php} can do an insertion inside that table and the value passed is not validated:
\lstinputlisting[firstline=27,lastline=28,caption=ManageAssignments.php store of \textit{assignment[5]}]{/var/www/sm/ManageAssignments.php}
\clearpage
Later on, the injected value can be read from the \emph{ViewAssignments.php} page and no validation is done:
\lstinputlisting[firstline=43,lastline=47,caption=ViewAssignments.php load query for \textit{assignment[5]}]{/var/www/sm/ViewAssignments.php}
\lstinputlisting[firstline=59,lastline=66,caption=ViewAssignments.php variable read of \textit{assignment[5]}]{/var/www/sm/ViewAssignments.php}
\end{homeworkProblem}
\clearpage

%----------------------------------------------------------------------------------------
%	VULNERABILITY 92
%----------------------------------------------------------------------------------------
\begin{homeworkProblem}{Vulnerability 92}
\subsection{Brief Analysis}
%\begin{center}
%	\includegraphics[scale=0.15]{/home/ashen/Scaricati/pixy/graphs/indexResults/92.jpg}
%\end{center}
\begin{center}
	File: ManageSchoolInfo.php
	\begin{table}[h]
	%caption{Example table}
		\begin{center}
    		\begin{tabular}{ | c | c | c |}
    			\hline
    			\textbf{VARIABLE} & \textbf{RESULT} \\ \hline
  				page & false positive \\ \hline
  				page2 & false positive \\ \hline
  				numperiods & false positive \\ \hline
  				numsemesters & false positive \\ \hline
  				phone & false positive \\ \hline
  				address & positive \\ \hline
  				schoolname & false positive \\ \hline
    		\end{tabular}
    	\end{center}
   \end{table}
\end{center}
\subsection{Explanation}
The analysis of the section \emph{Vulnerabilities\textsuperscript{(*)}} can also fit for \textit{page} and \textit{page2}.
Moreover, \emph{Vulnerabilities 2,3,4,6,10,53} explains the result over \textit{schoolname}.
\subsubsection{numperiods,numsemesters}
\lstinputlisting[firstline=14,lastline=22,caption=ManageSchoolInfo.php load of \textit{numperiods} and \textit{numsemesters}]{/var/www/sm/ManageSchoolInfo.php}
The load of the two values is not validated and that's why the software highlight the case, moreover we have a not validated update over the table:
\lstinputlisting[firstline=11,lastline=11,caption=header.php store of \textit{numperiods}-\textit{numsemesters}-\textit{phone}-\textit{address}]{/var/www/sm/header.php}
\clearpage
As happened for \emph{coursename} on \textit{Vulnerabilities 30,31,207}, the database schema tell us that no injection is possile, because the interested columns are set as int(3):
\lstinputlisting[firstline=175,lastline=190,caption=Sql schema for \textit{schoolinfo},language=Sql,label=dbsi]{/var/www/sm/SchoolMate.sql}
\subsubsection{phone}
\lstinputlisting[firstline=9,lastline=12,caption=ManageSchoolInfo.php load of \textit{phone}]{/var/www/sm/ManageSchoolInfo.php}
The load works as for the two field above, and as in that case, the result can be addressed as a \emph{false positive} thanks to the database schema (\textit{Listing \ref{dbsi}}). In this case the column has type \textit{varchar(14)}, which is more sensitive than int, but the size prevent any possible injection, because, as said in \textit{Vulnerabilities 30,31,207}, the smaller xss that we can apply - even if useless - is about 26 chars long.
\subsubsection{address}
\lstinputlisting[firstline=4,lastline=7,caption=ManageSchoolInfo.php load of \textit{address}]{/var/www/sm/ManageSchoolInfo.php}
This time the database schema cannot help us, because the field type is \textit{varchar(50)}, so an injection is possible. However it seems that the only page which reads from that field is the page itself - ManageSchoolInfo.php -, which puts the content as value in a form field. Still I will consider it as a positive result, because if in a future deployment the value will be displayed in an another page, the vulnerability will became exploitable.
\clearpage
\subsection{Testing Code}
\subsubsection{address}
\lstinputlisting[firstline=22,lastline=32,language=Java,caption=jwebunit test code for \textit{address}]{/home/ashen/workspace/TestingFramework/src/tests/xss.java}
\end{homeworkProblem}
\clearpage

%----------------------------------------------------------------------------------------
%	VULNERABILITY 105
%----------------------------------------------------------------------------------------

%% arrivato qui!! sistemare porzione di codice java!
\begin{homeworkProblem}{Vulnerability 105}
\subsection{Brief Analysis}
\begin{center}
	\includegraphics[scale=0.20]{/home/ashen/Scaricati/pixy/graphs/indexResults/105.jpg}
\end{center}
\begin{center}
	File: Login.php
	\begin{table}[h]
	%caption{Example table}
		\begin{center}
    		\begin{tabular}{ | c | c | c |}
    			\hline
    			\textbf{VARIABLE} & \textbf{RESULT} \\ \hline
  				message & positive \\ \hline
  				page & positive \\ \hline
    		\end{tabular}
    	\end{center}
   \end{table}
\end{center}
\subsection{Explanation}
The analysis of the section \emph{Vulnerabilities\textsuperscript{(*)}} can fit for \textit{page}.
\subsubsection{message}
\lstinputlisting[firstline=8,lastline=10,caption=Login.php load of \textit{sitemessage}]{/var/www/sm/Login.php}
\lstinputlisting[firstline=11,lastline=11,caption=header.php store of \textit{sitemessage}]{/var/www/sm/header.php}
\end{homeworkProblem}
\clearpage

%----------------------------------------------------------------------------------------
%	VULNERABILITY 234
%----------------------------------------------------------------------------------------
\begin{homeworkProblem}{Vulnerability 234}
\subsection{Brief Analysis}
\begin{center}
	\includegraphics[scale=0.20]{/home/ashen/Scaricati/pixy/graphs/indexResults/234.jpg}
\end{center}
\begin{center}
	File: ManageSemesters.php
	\begin{table}[h]
	%caption{Example table}
		\begin{center}
    		\begin{tabular}{ | c | c | c |}
    			\hline
    			\textbf{VARIABLE} & \textbf{RESULT} \\ \hline
  				term & positive \\ \hline
    		\end{tabular}
    	\end{center}
   \end{table}
\end{center}
\clearpage
\subsection{Explanation}
\lstinputlisting[firstline=132,lastline=133]{/var/www/sm/ManageSemesters.php}
\lstinputlisting[firstline=17,lastline=18]{/var/www/sm/ManageTerms.php}
\end{homeworkProblem}
\clearpage

%----------------------------------------------------------------------------------------
%	VULNERABILITY 269
%----------------------------------------------------------------------------------------
\begin{homeworkProblem}{Vulnerability 269}
\subsection{Brief Analysis}
\begin{center}
	\includegraphics[scale=0.20]{/home/ashen/Scaricati/pixy/graphs/indexResults/269.jpg}
\end{center}
\begin{center}
	File: AddClass.php
	\begin{table}[h]
	%caption{Example table}
		\begin{center}
    		\begin{tabular}{ | c | c | c |}
    			\hline
    			\textbf{VARIABLE} & \textbf{RESULT} \\ \hline
  				page & false positive \\ \hline
  				page2 & false positive \\ \hline
  				fullyear & false positive \\ \hline
    		\end{tabular}
    	\end{center}
   \end{table}
\end{center}
\subsection{Explanation}
The analysis of the section \emph{Vulnerabilities\textsuperscript{(*)}} can also fit for \textit{page} and \textit{page2}.
\subsubsection{fullyear}
The parameter is just used to display a different form of insertion of the class, so no xss is possible here.
\end{homeworkProblem}
\clearpage

%----------------------------------------------------------------------------------------
%	VULNERABILITY 321
%----------------------------------------------------------------------------------------
\begin{homeworkProblem}{Vulnerability 321}
\subsection{Brief Analysis}
\begin{center}
	\includegraphics[scale=0.25]{/home/ashen/Scaricati/pixy/graphs/indexResults/321.jpg}
\end{center}
\begin{center}
	File: ReportCards.php
	\begin{table}[h]
	%caption{Example table}
		\begin{center}
    		\begin{tabular}{ | c | c | c |}
    			\hline
    			\textbf{VARIABLE} & \textbf{RESULT} \\ \hline
  				data & positive \\ \hline
    		\end{tabular}
    	\end{center}
   \end{table}
\end{center}
\subsection{Explanation}
\lstinputlisting[firstline=205,lastline=207]{/var/www/sm/ReportCards.php}
\lstinputlisting[firstline=430,lastline=430]{/var/www/sm/ReportCards.php}
As long as seen at \emph{Vulnerabilities 30,31,207}, \textit{coursename} can be a injected with malicious strings which can lead to an xss vulnerability. In this case the pdf generated can contain such malicious string.
\end{homeworkProblem}
\clearpage
\end{document}